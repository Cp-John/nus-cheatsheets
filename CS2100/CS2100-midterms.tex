% !TEX TS-program = xelatex
\documentclass[10pt,landscape,a4paper]{article}
%\usepackage[utf8]{inputenc}
%\usepackage[ngerman]{babel}
\usepackage[normalem]{ulem}
\usepackage{tikz}
\usetikzlibrary{shapes,positioning,arrows,fit,calc,graphs,graphs.standard}
\usepackage[nosf]{kpfonts}
\usepackage[t1]{sourcesanspro}
%\usepackage[lf]{MyriadPro}
%\usepackage[lf,minionint]{MinionPro}
\usepackage{multicol}
\usepackage{wrapfig}
\usepackage[top=0mm,bottom=1mm,left=0mm,right=1mm]{geometry}
\usepackage[framemethod=tikz]{mdframed}
\usepackage{microtype}
%\usepackage{physics}
\usepackage{tabularx}
\usepackage{hhline}
\usepackage{makecell}
\usepackage{mathtools}

\usepackage{listings}
\usepackage{mips}
\lstset{language=[mips]Assembler}

\DeclarePairedDelimiter{\ceil}{\lceil}{\rceil}

\newcommand\codeblue[1]{\textcolor{blue}{\code{#1}}}

\usepackage{lastpage}
\usepackage{datetime}
\yyyymmdddate
\renewcommand{\dateseparator}{-}
\let\bar\overline

\definecolor{myblue}{cmyk}{1,.72,0,.38}

\def\firstcircle{(0,0) circle (1.5cm)}
\def\secondcircle{(0:2cm) circle (1.5cm)}

\colorlet{circle edge}{myblue}
\colorlet{circle area}{myblue!5}

\tikzset{filled/.style={fill=circle area, draw=circle edge, thick},
    outline/.style={draw=circle edge, thick}}

\pgfdeclarelayer{background}
\pgfsetlayers{background,main}

%\everymath\expandafter{\the\everymath \color{myblue}}
%\everydisplay\expandafter{\the\everydisplay \color{myblue}}


\renewcommand{\baselinestretch}{.8}
\pagestyle{empty}

\global\mdfdefinestyle{header}{%
linecolor=gray,linewidth=1pt,%
leftmargin=0mm,rightmargin=0mm,skipbelow=0mm,skipabove=0mm,
}

\newcommand{\header}{
\begin{mdframed}[style=header]
\footnotesize
\sffamily
CS2100 Midterms Cheatsheet v1.2\\(\today)\\
by~Julius Putra Tanu Setiaji,~page~\thepage~of~\pageref{LastPage}
\end{mdframed}
}
%\usepackage{chngcntr}

\usepackage{verbatim}

\usepackage{etoolbox}
\makeatletter
\preto{\@verbatim}{\topsep=0pt \partopsep=0pt }
\makeatother

\counterwithin*{equation}{section}
\counterwithin*{equation}{subsection}
\usepackage{enumitem}
\newlist{legal}{enumerate}{10}
\setlist[legal]{label*=\arabic*.,leftmargin=2.5mm}
\setlist[itemize]{leftmargin=3mm}
\setlist{nosep}
\usepackage{minted}

\def\code#1{\texttt{#1}}

\newenvironment{descitemize} % a mixture of description and itemize
{\begin{description}[leftmargin=*,before=\let\makelabel\descitemlabel]}
	{\end{description}}

\newcommand{\descitemlabel}[1]{%
	\textbullet\ \textbf{#1}%
}
\makeatletter



\renewcommand{\section}{\@startsection{section}{1}{0mm}%
                                {.2ex}%
                                {.2ex}%x
                                {\color{myblue}\sffamily\small\bfseries}}
\renewcommand{\subsection}{\@startsection{subsection}{1}{0mm}%
                                {.2ex}%
                                {.2ex}%x
                                {\sffamily\bfseries}}
\renewcommand{\subsubsection}{\@startsection{subsubsection}{1}{0mm}%
                                {.2ex}%
                                {.2ex}%x
                                {\rmfamily\bfseries}}



\def\multi@column@out{%
   \ifnum\outputpenalty <-\@M
   \speci@ls \else
   \ifvoid\colbreak@box\else
     \mult@info\@ne{Re-adding forced
               break(s) for splitting}%
     \setbox\@cclv\vbox{%
        \unvbox\colbreak@box
        \penalty-\@Mv\unvbox\@cclv}%
   \fi
   \splittopskip\topskip
   \splitmaxdepth\maxdepth
   \dimen@\@colroom
   \divide\skip\footins\col@number
   \ifvoid\footins \else
      \leave@mult@footins
   \fi
   \let\ifshr@kingsaved\ifshr@king
   \ifvbox \@kludgeins
     \advance \dimen@ -\ht\@kludgeins
     \ifdim \wd\@kludgeins>\z@
        \shr@nkingtrue
     \fi
   \fi
   \process@cols\mult@gfirstbox{%
%%%%% START CHANGE
\ifnum\count@=\numexpr\mult@rightbox+2\relax
          \setbox\count@\vsplit\@cclv to \dimexpr \dimen@-1cm\relax
\setbox\count@\vbox to \dimen@{\vbox to 1cm{\header}\unvbox\count@\vss}%
\else
      \setbox\count@\vsplit\@cclv to \dimen@
\fi
%%%%% END CHANGE
            \set@keptmarks
            \setbox\count@
                 \vbox to\dimen@
                  {\unvbox\count@
                   \remove@discardable@items
                   \ifshr@nking\vfill\fi}%
           }%
   \setbox\mult@rightbox
       \vsplit\@cclv to\dimen@
   \set@keptmarks
   \setbox\mult@rightbox\vbox to\dimen@
          {\unvbox\mult@rightbox
           \remove@discardable@items
           \ifshr@nking\vfill\fi}%
   \let\ifshr@king\ifshr@kingsaved
   \ifvoid\@cclv \else
       \unvbox\@cclv
       \ifnum\outputpenalty=\@M
       \else
          \penalty\outputpenalty
       \fi
       \ifvoid\footins\else
         \PackageWarning{multicol}%
          {I moved some lines to
           the next page.\MessageBreak
           Footnotes on page
           \thepage\space might be wrong}%
       \fi
       \ifnum \c@tracingmulticols>\thr@@
                    \hrule\allowbreak \fi
   \fi
   \ifx\@empty\kept@firstmark
      \let\firstmark\kept@topmark
      \let\botmark\kept@topmark
   \else
      \let\firstmark\kept@firstmark
      \let\botmark\kept@botmark
   \fi
   \let\topmark\kept@topmark
   \mult@info\tw@
        {Use kept top mark:\MessageBreak
          \meaning\kept@topmark
         \MessageBreak
         Use kept first mark:\MessageBreak
          \meaning\kept@firstmark
        \MessageBreak
         Use kept bot mark:\MessageBreak
          \meaning\kept@botmark
        \MessageBreak
         Produce first mark:\MessageBreak
          \meaning\firstmark
        \MessageBreak
        Produce bot mark:\MessageBreak
          \meaning\botmark
         \@gobbletwo}%
   \setbox\@cclv\vbox{\unvbox\partial@page
                      \page@sofar}%
   \@makecol\@outputpage
     \global\let\kept@topmark\botmark
     \global\let\kept@firstmark\@empty
     \global\let\kept@botmark\@empty
     \mult@info\tw@
        {(Re)Init top mark:\MessageBreak
         \meaning\kept@topmark
         \@gobbletwo}%
   \global\@colroom\@colht
   \global \@mparbottom \z@
   \process@deferreds
   \@whilesw\if@fcolmade\fi{\@outputpage
      \global\@colroom\@colht
      \process@deferreds}%
   \mult@info\@ne
     {Colroom:\MessageBreak
      \the\@colht\space
              after float space removed
              = \the\@colroom \@gobble}%
    \set@mult@vsize \global
  \fi}
\global\let\tikz@ensure@dollar@catcode=\relax

\def\mathcolor#1#{\@mathcolor{#1}}
\def\@mathcolor#1#2#3{%
	\protect\leavevmode
	\begingroup
	\color#1{#2}#3%
	\endgroup
}

\makeatother
\setlength{\parindent}{0pt}

\setminted{tabsize=2, breaklines}
% Remove belowskip of minted
\setlength\partopsep{-\topsep}


\newcolumntype{a}{>{\hsize=1.5\hsize}X}
\newcolumntype{b}{>{\hsize=.25\hsize}X}

\setlength\columnsep{1.5pt}
\setlength\columnseprule{0.1pt}
\begin{document}
\setlength{\abovedisplayskip}{0pt}
	\setlength{\belowdisplayskip}{0pt}

\scriptsize
\begin{multicols*}{3}
	\raggedcolumns
	\section{C Programming Language}
	\subsection{Data Types}
	\newcolumntype{a}{>{\hsize=1.5\hsize}X}
	\newcolumntype{b}{>{\hsize=.25\hsize}X}
	\begin{tabularx}{\columnwidth}{|b|b|a|}
		\hline
		\textbf{Type} & \textbf{\texttt{sizeof}} & \textbf{range} \\
		\hline
		\texttt{int} & 4 bytes & \makecell[l]{2s complement, thus $(-2^{31})$ to $(2^{31}-1)$ \\ OR (-2,147,483,648 to 2,147,483,647)} \\ \hline
		\texttt{float} & 4 bytes & 1-bit sign, 8-bit exponent (excess-127), 23-bit mantissa \\ \hline
		\texttt{double} & 8 bytes & 1-bit sign, 11-bit exponent (excess-1023), 52-bit mantissa \\ \hline
		\texttt{char} & 1 byte & ASCII (7 bits + 1 parity bit), A is 100 0001 \\ \hline
	\end{tabularx}
	Note that for mantissa there is an implicit leading bit 1
	\subsection{Format Specifiers}
	\begin{tabular}{|l|l|l|}
		\hline
		& \textbf{Type} & \textbf{fn} \\
		\hline
		\texttt{\%c} & \texttt{char} & \texttt{printf/scanf} \\ \hline
		\texttt{\%d} & \texttt{char} & \texttt{printf/scanf} \\ \hline
		\texttt{\%f} & \texttt{float}/\texttt{double} & \texttt{printf} \\ \hline
	\end{tabular}
	\begin{tabular}{|l|l|l|}
		\hline
		& \textbf{Type} & \textbf{fn} \\
		\hline
		\texttt{\%f} & \texttt{float} & \texttt{scanf} \\ \hline
		\texttt{\%lf} & \texttt{double} & \texttt{scanf} \\ \hline
		\texttt{\%p} & pointers & \texttt{printf} \\ \hline
	\end{tabular}
	\subsection{Escape Sequences}
	\begin{tabular}{|l|l|}
		\hline
		& \textbf{Meaning} \\
		\hline
		\texttt{\textbackslash n} & new line \\ \hline
		\texttt{\textbackslash t} & tab \\ \hline
	\end{tabular}
	\begin{tabular}{|l|l|}
		\hline
		& \textbf{Meaning} \\
		\hline
		\texttt{\textbackslash "} & double-quote " \\ \hline
		\texttt{\%\%} & percent \% \\ \hline
	\end{tabular}
	\subsection{Misc}
		\begin{itemize}
			\item Short-circuit evaluation
		\end{itemize}
	\section{Numbering Systems}
	\subsection{Data Representation}
		\begin{itemize}
			\item 1 byte = 8 bits
			\item $n$ bits can represent up to $2^n$ values. Thus, to present $m$ values, $\ceil{\log_2m}$ is required
		\end{itemize}
	\subsection{Decimal to Binary Conversion}
		\begin{itemize}
			\item For whole numbers: repeated division-by-2 (look at remainder)
			\item For fractions: repeated multiplication-by-2 (look at "quotient")
		\end{itemize}
	\subsection{Representation of Signed Binary Numbers}
		\newcolumntype{c}{>{\hsize=0.45\hsize}X}
		\newcolumntype{d}{>{\hsize=0.6\hsize}X}
		\newcolumntype{e}{>{\hsize=0.55\hsize}X}
		\newcolumntype{f}{>{\hsize=0.4\hsize}X}
		\begin{tabularx}{\columnwidth}{|c|d|e|f|}
			\hline
			 & Negation & Range & Zeroes \\
			\hline
			Sign-and-Magnitude & invert the sign bit (leading bit) & $-(2^{n-1} - 1)$ to $2^{n-1} - 1$ & $+0_{10}$ and $-0_{10}$ \\ \hline
			1s Complement & invert all the bits & $-(2^{n-1} - 1)$ to $2^{n-1} - 1$ & $+0_{10}$ and $-0_{10}$ \\ \hline
			2s Complement & invert all the bits, then add 1 & $-2^{n-1}$ to $2^{n-1}-1$ & $+0_{10}$ \\ \hline
		\end{tabularx}
		For all of the above, the MSB (Most Significant Bit) represents sign.
		\subsubsection{Sign-and-Magnitude}
			\begin{itemize}
				\item Range (8-bit): $(1111$ $1111)$ to $(0111$ $1111) = -127_{10}$ to $+127_{10}$
				\item Zeroes: $0000$ $0000 = +0_{10}$ and $1000$ $0000 = -0_{10}$
				\item e.g. $(\mathcolor{red}{0}011$ $0100)_{sm} = +011$ $0100_2 = +52_{10}$, $(\mathcolor{red}{1}001$ $0011)_{sm} = -(001$ $0011)_2 = -(19)_{10}$
			\end{itemize}
		\subsubsection{1s Complement (Diminished Radix)}
			\begin{itemize}
				\item Negation: $-x=2^n -x-1$
				\item Range (8-bit): $(1000$ $0000)$ to $(0111$ $1111) = -127_{10}$ to $+127_{10}$
				\item Zeroes: $(0000$ $0000) = +0_{10}$ and $(1111$ $1111) = -0_{10}$
				\item e.g. $(0000$ $1110)_{1s} = (0000$ $1110)_2 = (14)_{10}$,
				$(1111$ $0001)_{1s} = -(0000$ $1110)_2 = -(14)_{10}$
			\end{itemize}
		\subsubsection{2s Complement (Radix complement)}
			\begin{itemize}
				\item Negation: $-x = 2^n - x$
				\item Range (8-bit): $(1000$ $0000) = -128_{10}$ to $(0111$ $1111) = +127_{10}$
				\item Zero: $(0000$ $0000) = +0_{10}$
				\item e.g. $(0000$ $1110)_{2s} = (0000$ $1110)_2 = (14)_{10}$,  $(1111$ $0010)_{2s} = -(0000$ $1110)_2 = -(14)_{10}$
			\end{itemize}
		\subsubsection{Excess-k}
			\begin{itemize}
				\item Also known as offset binary. Use 0000 to represent $-k$ (lowest number possible)
				\item For unsigned, with $n$-bit number, $k=2^{n-1}-1$
			\end{itemize}
		\subsubsection{Comparison}
		\newcolumntype{g}{>{\hsize=0.15\hsize}X}
		\newcolumntype{h}{>{\hsize=0.4\hsize}X}
		\begin{tabularx}{\columnwidth}{|g|h|h|h|h|g|}
			\hline
			\textbf{Value} & \textbf{Sign-and-Magnitude} & \textbf{1s Complement} & \textbf{2s Complement} & \textbf{Excess-8} & \textbf{Value} \\
			\hline
			+7 & 0111 & 0111 & 0111 & 1111 & +7 \\ \hline
			+6 & 0110 & 0110 & 0110 & 1110 & +6 \\ \hline
			+5 & 0101 & 0101 & 0101 & 1101 & +5 \\ \hline
			+4 & 0100 & 0100 & 0100 & 1100 & +4 \\ \hline
			+3 & 0011 & 0011 & 0011 & 1011 & +3 \\ \hline
			+2 & 0010 & 0010 & 0010 & 1010 & +2 \\ \hline
			+1 & 0001 & 0001 & 0001 & 1001 & +1 \\ \hline
			+0 & 0000 & 0000 & 0000 & 1000 & +0 \\ \hline
			-0 & 1000 & 1111 & -    & -    & -0 \\ \hline
			-1 & 1001 & 1110 & 1111 & 0111 & -1 \\ \hline
			-2 & 1010 & 1101 & 1110 & 0110 & -2 \\ \hline
			-3 & 1011 & 1100 & 1101 & 0101 & -3 \\ \hline
			-4 & 1100 & 1011 & 1100 & 0100 & -4 \\ \hline
			-5 & 1101 & 1010 & 1011 & 0011 & -5 \\ \hline
			-6 & 1110 & 1001 & 1010 & 0010 & -6 \\ \hline
			-7 & 1111 & 1000 & 1001 & 0001 & -7 \\ \hline
			-8 & -    & -    & 1000 & 0000 & -8 \\ \hline
		\end{tabularx}
			
		\subsection{Operation on binary numbers}
		Algorithm for \textbf{Subtraction}: $A - B = A + (-B)$\\
		Algorithm for \textbf{Overflow Check}: if MSB of first and second are the same, then MSB of resulting numbers must be the same too.
		\subsubsection{2s Complement on Addition}
		Algorithm: (1) Perform binary addition. (2) Ignore the carry out of the MSB. (3) Check for overflow.
\begin{verbatim}
Example, 2s Complement 4-bit
+3  0011                -2    1110                -3     1101
+4  0100                -6    1010                -6     1010
--- -----               ---   -----               ---    -----
+7  0111 (No overflow)  -8 (1)1000 (No overflow)  -9  (1)0111 (Overflow!)
\end{verbatim}
		\subsubsection{1s Complement on Addition}
		Algorithm: (1) Perform binary addition. (2) If there is carry out of the MSB, add 1 to the result. (3) Check for overflow.
\begin{verbatim}
Example, 1s Complement 4-bit
+3  0011                -2    1101                -3     1100
+4  0100                -5    1010                -6     1001
--- -----               ---   -----               ---    -----
+7  0111 (No overflow)  -7 (1)0111                -9  (1)0101
                                 1                          1
                              -----                      -----
                              1000 (No overflow)         0110 (Overflow!)
\end{verbatim}
		\subsection{Floating Point}
		\begin{itemize}
			\item Single precision 32 bits: 1-bit sign, 8-bit exponent (excess-127), 23-bit mantissa
			\item Double precision 64 bits: 1-bit sign, 11-bit exponent (excess-1023), 52-bit mantissa
			\item e.g. $-6.5_{10} = -110.1_{2} = -1.101_{2} \times 2^2$, Exponent (excess-127) $= 2 + 127 = 129 = 1000$ $0001_2$
		\end{itemize}
			
\begin{verbatim}
Sign Exponent          Mantissa
 1   10000001 10100000000000000000000
\end{verbatim}
Hence, $1100$ $0000$ $1101$ $0000$ $0000$ $0000$ $0000$ $0000_2 = C0D0 0000_{16}$\\
(as \texttt{float}$ = -6.5$, as \texttt{int}$ = -1,060,110,336$)
	\section{Pointers and Functions}
	\subsection{Pointers}
	\begin{itemize}
		\item New unary operators: \texttt{*} and \texttt{\&}
		\item Convention: \mintinline{C}{int *abc;} AND \mintinline{C}{void f(int *);}
	\end{itemize}
	\subsection{Functions}
	\begin{itemize}
		\item Function prototype (just the type of its parameters): e.g. \mintinline{C}{void g(int, int);}
		\item Variable scoping: by \textbf{functions}
	\end{itemize}
	\section{Arrays, Strings, Structures}
	\subsection{Arrays}
		\begin{itemize}
			\item Array is a homogeneous collection of data, occupying contiguous memory locations.
			\item When initialised with fewer values than elements, the rest are initialised as 0 (for \mintinline{C}{int}).
			\item Equivalence: value: \mintinline{C}{*(arr+2) == arr[2]} and memory location: \mintinline{C}{arr + 2 == &arr[2]}
		\end{itemize}
	\subsection{String}
	\begin{itemize}
		\item An array of characters, terminated by a null character \mintinline{C}{'\0'} (ASCII value: 0	)
		\item Initialising: \mintinline{C}{char str[4] = "egg";} or \mintinline{C}{char str[4] = {'e', 'g', 'g', '\0'};}
		\item Read from stdin: \mintinline{C}{fgets(str, size, stdin); // reads until (size - 1) or '\n'} and \mintinline{C}{scanf("%s", str); // reads until whitespace} \\
		(note that \mintinline{C}{fgets} also reads in \mintinline{C}{'\n'})
		\item Print to stdout: \mintinline{C}{puts(str);} which is equivalent to \mintinline{C}{printf("%s\n", str);}
		\item String functions:
		\begin{itemize}
			\item \mintinline{C}{strlen(s)}: returns the no of chars in \mintinline{C}{s}
			\item \mintinline{C}{strcmp(s1, s2)}: compare ASCII values of corresponding characters, returns $\mathbb{Z}^+$ if s1 is lexographically greater, $0$ if equal, $\mathbb{Z}^-$ otherwise
			\item \mintinline{C}{strncmp(s1, s2, n)}: compare first $n$ chars of s1 and s2
			\item \mintinline{C}{strcpy(s1, s2)}: copy the string pointed by s2 into array pointed by s1, returns s1. E.g. \mintinline{C}{char s[4]; strcpy(s, "asdfgh"); // s == {'a', 's', 'd', '\0'};}
			\item \mintinline{C}{strncpt(s1, s2, n)}: copy the first $n$ chars of string pointed by s2 to s1
		\end{itemize}
	\end{itemize}
	\subsection{Structures}
	\begin{itemize}
		\item Structures allow grouping of heterogeneous members of different types.
		\item Assignment  \mintinline{C}{result2 = result1;} copies the entire structure.
		\item Passing structure to function: the entire structure is copied.
		\item Alternatively, to change original structure, one can use pointer. Syntactic sugar: \mintinline{C}{(*player_ptr).name == player_ptr->name;}
		\item E.g.: \begin{minted}{C}
typedef struct {
	int day, month, year;
} date_t;
typedef struct {
	int stuNum;
	date_t birthday;
} student_t;
student_t s1 = {1049858, {31, 12, 2020}}; // s1.birthday.month == 2020
\end{minted}
	\end{itemize}
	\section{C for Hardware Programming}
	\subsection{Code Compilation Process}
	C Program (.c) -> \textbf{Preprocessor} -> Preprocessed code (.i) -> \textbf{Compiler} -> Assembly code (.asm) -> \textbf{Assembler} -> Object code (.o) -> \textbf{Linker} -> Executable (.hex)
	\section{MIPS}
	\subsection{Loading a 32-bit constant into a register}
		\begin{enumerate} 
			\item Use \texttt{lui} to set the upper 16-bit: \lstinline|lui $t0, 0xAAAA|
			\item Use \texttt{ori} to set the lower-order bits: \lstinline|ori $t0, $t0, 0xF0F0|
		\end{enumerate}
	\subsection{Memory Organisation}
		\begin{itemize}
			\item Each address contains 1 byte = 8 bit of content.
			\item Memory addresses are 32-bit long ($2^{30}$ memory words).
			\item 32 registers, each 4-byte long.	Each word is also 4-byte long. 
		\end{itemize}
	\subsection{MIPS Instruction Classification}
		\subsubsection{R-format}
			\begin{itemize}
				\item \texttt{op \$rd, \$rs, \$rt}
				\item \texttt{sll \$rd, \$rt, shamt} (rs = 0)
			\end{itemize}
		\subsubsection{I-format}
			\begin{itemize}
				\item \texttt{op \$rt, \$rs, Immediate}
				\item Displacement address: offset from address in rs
				\item PC-relative address: no of instructions from next instruction $PC = (PC + immediate) \times 4$
			\end{itemize}		
		\subsubsection{J-format}
			\begin{itemize}
				\item \texttt{op Immediate}
				\item pseudo-direct address: remove last 2 bit (since word-aligned, by default the 2 least significant bits are 00) and 4 most significant bits (always the same as instruction address).
				\item eg \texttt{xxxx00001111000011110000111100\sout{00}}, immediate is \texttt{00001111000011110000111100}
			\end{itemize}
		\section{Instruction Set Architecture}
			For modern processors: \textbf{General-Purpose Register} (GPR) is most common. \textbf{RISC} typically uses \textbf{Register-Register (Load/Store)} design, e.g. MIPS, ARM. \textbf{CISC} use a mixture of Register-Register and Register-Memory, e.g. IA32
			\subsection{Data Storage}
				\begin{descitemize}
					\item [Stack architecture]: Operands are implicitly on top of the stack.
					\item [Accumulator architecture]: One operand is implicitly in the accumulator (a special register)
					\item [General-purpose register architecture]: only explicit operands
					\begin{descitemize}
						\item [Register-memory architecture]: one operand in memory.
						\item [Register-register (or load store) architecture]
					\end{descitemize}
					\item [Memory-memory architecture]: all operands in memory.
				\end{descitemize}
			\subsection{Memory Addressing Modes}
				\begin{descitemize}
					\item [Endianness]:
					\begin{descitemize}
						\item [Big-endian]: Most significant \textbf{byte} stored in lowest address
						\item [Little-endian]: Least significant \textbf{byte} stored in lowest address ("reverse-order")
					\end{descitemize}
					\item [Addressing modes]: in MIPS, only 3: \textbf{Register} \lstinline|add $t1, $t2, $t3|, \textbf{Immediate} \lstinline|addi $t1, $t2, 98|, \textbf{Displacement} \lstinline|lw $t1, 20($t2)|
				\end{descitemize}
			\subsection{Operations in the instruction set}
				Amdahl's law: make common cases fast. Optimise frequently used instructions (\textbf{Load}: 22\%, \textbf{Conditional Branch}: 20\%, \textbf{Compare} 16\%, \textbf{Store}: 12\%)
			\subsection{Instruction Formats}
			\begin{descitemize}
				\item [Instruction Length]:
				\begin{descitemize}
					\item [Variable-length instructions]: Require multi-step fetch and decode. Allow for a more flexible (but complex) and compact instruction set.
					\item [Fixed-length instructions]: used in most RISC, e.g. MIPS instructions are 4-bytes long. Allow for easy fetch and decode, simplify pipelining and parallelism. Instruction bits are scarce.
					\item [Hybrid instructions]: a mix of variable- and fixed-length instructions.
				\end{descitemize}
				\item [Instruction Fields]: \textbf{opcode} (unique code to specify the desired operation) and \textbf{operands} (zero or more additional information needed for the operation)
			\end{descitemize}
			\subsection{Encoding the Instruction Set}
			\begin{descitemize}
				\item [Expanding Opcode] scheme:
				\begin{descitemize}
					\item E.g. \textbf{Type-A}: \textbf{6-bit} opcode, \textbf{Type-B}: \textbf{11-bits} opcode. Max no of instructions = $1 + (2^6 - 1)\times 2^5 = 2017$\\
					(1 Type-A instruction, Type-B "steals" [$2^6-1$] opcodes from Type-A to prefix, each prefix having [$2^{11-6} = 2^5$] opcodes)
				\end{descitemize}
			\end{descitemize}
		\section{Datapath}
			\subsection{Instruction Execution Cycle}
			For MIPS: (1)Fetch (2)Decode \& Operand Fetch (3)ALU (4)Memory Access (5)Result Write
			\begin{descitemize}
				\item [Fetch]: Get instruction from memory, address is in Program Counter (PC) Register
				\item [Decode]: Find out the operation required
				\item [Operand Fetch]: Get operand(s) needed for operation
				\item [Execute]: Perform the required oepration
				\item [Result Write (Store)]: Store the result of the operation
			\end{descitemize}
			\subsection{Elements}
				\begin{descitemize}
					\item [Adder] \textbf{Input}: two 32-bit numbers, \textbf{Output}: sum of input numbers
					\item [Register File] \textbf{Input}: three 5-bit: Read register 1, Read register 2, Write register; 32-bit Write data, \textbf{Output}: two 32-bit Read data 1, Read data 2; \textbf{Control}: 1-bit RegWrite (1 = write)
					\item [Multiplexer] \textbf{Input}: $n$ lines of same width, \textbf{Control}: $m$ bits where $n=2^m$, \textbf{Output}: Select $i^{th}$ input line if control = $i$
					\item [Arithmetic Logic Unit]: \textbf{Input}: two 32-bit numbers, \textbf{Control}: 4-bit to decide the particular operation, \textbf{Output}: 32-bit ALU result, 1-bit isZero? \\
					\begin{tabular}{|l|l|}
						\hline
						\textbf{ALUcontrol} & \textbf{Function} \\ \hline
						\texttt{0000} & AND \\
						\texttt{0001} & OR \\
						\texttt{0010} & add \\
						\hline
					\end{tabular}
					\begin{tabular}{|l|l|}
						\hline
						\textbf{ALUcontrol} & \textbf{Function} \\ \hline
						\texttt{0110} & subtract \\
						\texttt{0111} & slt \\
						\texttt{1100} & NOR \\
						\hline
					\end{tabular}
					\item [Data Memory] \textbf{Input}: 32-bit memory address, 32-bit write data; \textbf{Control}: 1-bit MemWrite, 1-bit MemRead; \textbf{Output}: 32-bit ReadData
				\end{descitemize}
			
%		\begin{minted}{javascript}
		
%		\end{minted}
\end{multicols*}
\end{document}