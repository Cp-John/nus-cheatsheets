% !TEX TS-program = xelatex
\documentclass[10pt,landscape,a4paper]{article}
%\usepackage[utf8]{inputenc}
%\usepackage[ngerman]{babel}
\usepackage{tikz}
\usetikzlibrary{shapes,positioning,arrows,fit,calc,graphs,graphs.standard}
\usepackage[nosf]{kpfonts}
\usepackage[t1]{sourcesanspro}
%\usepackage[lf]{MyriadPro}
%\usepackage[lf,minionint]{MinionPro}
\usepackage{multicol}
\usepackage{wrapfig}
\usepackage[top=0mm,bottom=1mm,left=0mm,right=1mm]{geometry}
\usepackage[framemethod=tikz]{mdframed}
\usepackage{microtype}
\usepackage{lastpage}
%\usepackage{physics}
\usepackage{datetime}
\yyyymmdddate
\renewcommand{\dateseparator}{-}
\let\bar\overline

\definecolor{myblue}{cmyk}{1,.72,0,.38}

\def\firstcircle{(0,0) circle (1.5cm)}
\def\secondcircle{(0:2cm) circle (1.5cm)}

\colorlet{circle edge}{myblue}
\colorlet{circle area}{myblue!5}

\tikzset{filled/.style={fill=circle area, draw=circle edge, thick},
    outline/.style={draw=circle edge, thick}}

\pgfdeclarelayer{background}
\pgfsetlayers{background,main}

\everymath\expandafter{\the\everymath \color{myblue}}
%\everydisplay\expandafter{\the\everydisplay \color{myblue}}


\renewcommand{\baselinestretch}{.8}
\pagestyle{empty}

\global\mdfdefinestyle{header}{%
linecolor=gray,linewidth=1pt,%
leftmargin=0mm,rightmargin=0mm,skipbelow=0mm,skipabove=0mm,
}

\newcommand{\header}{
\begin{mdframed}[style=header]
\footnotesize
\sffamily
MA1521 Finals Cheatsheet v1.2 (\today)\\
by~Julius~Putra~Tanu~Setiaji,~page~\thepage~of~\pageref{LastPage}
\end{mdframed}
}
%\usepackage{chngcntr}

\usepackage{pgfplots}
\pgfplotsset{compat=1.8}
\counterwithin*{equation}{section}
\counterwithin*{equation}{subsection}
\usepackage{enumitem}
\newlist{legal}{enumerate}{10}
\setlist[legal]{label*=\arabic*.,leftmargin=4mm}
\setlist[itemize]{leftmargin=4mm}
\setlist{itemsep=0mm,topsep=1mm,leftmargin=4mm}
\newenvironment{descitemize} % a mixture of description and itemize
{\begin{description}[leftmargin=*,before=\let\makelabel\descitemlabel]}
	{\end{description}}

\newcommand{\descitemlabel}[1]{%
	\textbullet\ \textbf{#1}%
}
\makeatletter



\renewcommand{\section}{\@startsection{section}{1}{0mm}%
                                {.2ex}%
                                {.2ex}%x
                                {\color{myblue}\sffamily\small\bfseries}}
\renewcommand{\subsection}{\@startsection{subsection}{1}{0mm}%
                                {.2ex}%
                                {.2ex}%x
                                {\sffamily\bfseries}}



\def\multi@column@out{%
   \ifnum\outputpenalty <-\@M
   \speci@ls \else
   \ifvoid\colbreak@box\else
     \mult@info\@ne{Re-adding forced
               break(s) for splitting}%
     \setbox\@cclv\vbox{%
        \unvbox\colbreak@box
        \penalty-\@Mv\unvbox\@cclv}%
   \fi
   \splittopskip\topskip
   \splitmaxdepth\maxdepth
   \dimen@\@colroom
   \divide\skip\footins\col@number
   \ifvoid\footins \else
      \leave@mult@footins
   \fi
   \let\ifshr@kingsaved\ifshr@king
   \ifvbox \@kludgeins
     \advance \dimen@ -\ht\@kludgeins
     \ifdim \wd\@kludgeins>\z@
        \shr@nkingtrue
     \fi
   \fi
   \process@cols\mult@gfirstbox{%
%%%%% START CHANGE
\ifnum\count@=\numexpr\mult@rightbox+2\relax
          \setbox\count@\vsplit\@cclv to \dimexpr \dimen@-1cm\relax
\setbox\count@\vbox to \dimen@{\vbox to 1cm{\header}\unvbox\count@\vss}%
\else
      \setbox\count@\vsplit\@cclv to \dimen@
\fi
%%%%% END CHANGE
            \set@keptmarks
            \setbox\count@
                 \vbox to\dimen@
                  {\unvbox\count@
                   \remove@discardable@items
                   \ifshr@nking\vfill\fi}%
           }%
   \setbox\mult@rightbox
       \vsplit\@cclv to\dimen@
   \set@keptmarks
   \setbox\mult@rightbox\vbox to\dimen@
          {\unvbox\mult@rightbox
           \remove@discardable@items
           \ifshr@nking\vfill\fi}%
   \let\ifshr@king\ifshr@kingsaved
   \ifvoid\@cclv \else
       \unvbox\@cclv
       \ifnum\outputpenalty=\@M
       \else
          \penalty\outputpenalty
       \fi
       \ifvoid\footins\else
         \PackageWarning{multicol}%
          {I moved some lines to
           the next page.\MessageBreak
           Footnotes on page
           \thepage\space might be wrong}%
       \fi
       \ifnum \c@tracingmulticols>\thr@@
                    \hrule\allowbreak \fi
   \fi
   \ifx\@empty\kept@firstmark
      \let\firstmark\kept@topmark
      \let\botmark\kept@topmark
   \else
      \let\firstmark\kept@firstmark
      \let\botmark\kept@botmark
   \fi
   \let\topmark\kept@topmark
   \mult@info\tw@
        {Use kept top mark:\MessageBreak
          \meaning\kept@topmark
         \MessageBreak
         Use kept first mark:\MessageBreak
          \meaning\kept@firstmark
        \MessageBreak
         Use kept bot mark:\MessageBreak
          \meaning\kept@botmark
        \MessageBreak
         Produce first mark:\MessageBreak
          \meaning\firstmark
        \MessageBreak
        Produce bot mark:\MessageBreak
          \meaning\botmark
         \@gobbletwo}%
   \setbox\@cclv\vbox{\unvbox\partial@page
                      \page@sofar}%
   \@makecol\@outputpage
     \global\let\kept@topmark\botmark
     \global\let\kept@firstmark\@empty
     \global\let\kept@botmark\@empty
     \mult@info\tw@
        {(Re)Init top mark:\MessageBreak
         \meaning\kept@topmark
         \@gobbletwo}%
   \global\@colroom\@colht
   \global \@mparbottom \z@
   \process@deferreds
   \@whilesw\if@fcolmade\fi{\@outputpage
      \global\@colroom\@colht
      \process@deferreds}%
   \mult@info\@ne
     {Colroom:\MessageBreak
      \the\@colht\space
              after float space removed
              = \the\@colroom \@gobble}%
    \set@mult@vsize \global
  \fi}

\makeatother
\setlength{\parindent}{0pt}
\setlength\columnsep{1.5pt}
\setlength\columnseprule{0.1pt}
\begin{document}

\footnotesize
\begin{multicols*}{4}
	\raggedcolumns
	\section{Trigo Formulae}
	\begin{descitemize}
	\item $\sin^2 \theta + \cos^2 \theta = 1$, $\sin 2\theta = 2\sin \theta \cos \theta$
	\item $\sin(A \pm B) = \sin A\cos B \pm \sin B \cos A$
	\item $\cos(A \pm B) = \cos A \cos B \mp \sin A \sin B$
	\item $\tan(A \pm B) = \frac{\tan A \pm \tan B}{1 \mp \tan A \tan B}$, $\tan 2\theta = \frac{2\tan \theta}{1-\tan^2 \theta}$
	\item $\cos 2\theta = cos^2 \theta - sin^2 \theta = 2\cos ^2 \theta - 1 = 1 - 2\sin^2 \theta$
	\item $\sin P + \sin Q = 2\sin \frac{1}{2}(P+Q) \cos \frac{1}{2}(P-Q)$
	\item $\sin P - \sin Q = 2\cos \frac{1}{2}(P+Q) \sin\frac{1}{2}(P-Q)$
	\item $\cos P + \cos Q = 2\cos \frac{1}{2}(P+Q) \cos\frac{1}{2}(P-Q)$
	\item $\cos P - \cos Q = -2\sin \frac{1}{2}(P+Q) \sin\frac{1}{2}(P-Q)$
	\item $a^2 = b^2+c^2-2bc\cos\theta$ and $\frac{a}{\sin a} = \frac{b}{\sin b}$
	\end{descitemize}
	\section{Functions and Limits}
	\subsection*{Existence of Limits}
	$\lim\limits_{x\to a} f(x)$ only exists when:
	\begin{descitemize}
	\item $\lim\limits_{x\to a^-} f(x) = \lim\limits_{x\to a^+} f(x)$ (limit from left = right)
	\item For $a=\infty$ or $-\infty$, only if $f(x)$ \textbf{does not} oscillate
	\end{descitemize}
	\subsection*{Rules of Limits}
	\begin{enumerate}
	\item $\lim\limits_{x\to a} (f \pm g)(x) = \lim\limits_{x\to a}f(x) \pm \lim\limits_{x\to a} g(x)$
	\item $\lim\limits_{x\to a} f(x)g(x) = \lim\limits_{x\to a}f(x) \lim\limits_{x\to a}g(x)$
	\item $\lim\limits_{x\to a} \frac{f(x)}{g(x)} = \frac{\lim\limits_{x\to a} f(x)}{\lim\limits_{x\to a} g(x)}$ provided $\lim\limits_{x\to a} g(x) \neq 0$
	\item $\lim\limits_{x\to a} kf(x) = k \lim\limits_{x\to a} f(x)$
	\end{enumerate}
	\subsection*{Continuity}
	$f$ is continuous at point $a \Leftrightarrow \lim\limits_{x\to a} f(x) = f(a)$
	\subsection*{L'H\^opital's Rule}
	Suppose:
	\begin{enumerate}
	\item $f$ and $g$ are differentiable
	\item $f(a)=g(a)=0$
	\item $g'(x)\neq0$ for all $x \in I \setminus {a}$
	\end{enumerate}
	Then $\lim\limits_{x\to a}\frac{f(x)}{g(x)} = \lim\limits_{x\to a}\frac{f'(x)}{g'(x)}$\\
	\begin{descitemize}
	\item Use L'H\^opital's Rule for $\frac{0}{0}$ and $\frac{\infty}{\infty}$ forms.
	
	\item Common: $\lim\limits_{x\to\frac{\pi^-}{2}}(\sin x)^{\tan x} = \lim\limits_{x\to\frac{\pi^-}{2}} e^{\ln(\sin x)^{\tan x}}\\=e^{\lim\limits_{x\to\frac{\pi^-}{2}}\tan x\ln(\sin x)}=e^{\lim\limits_{x\to\frac{\pi^-}{2}}\frac{\ln(\sin x)}{\cot x}}$ (now in $\frac{0}{0}$ form)
	\item Tips:
	\begin{enumerate}
	\item Convert $0 \cdot \infty, \infty - \infty$ by algebra manip
	\item Convert $ 1^\infty, \infty^0, 0^0$ by first taking $\ln$
	\end{enumerate}
	\end{descitemize}
	
	\section{Derivative}
	The derivative of $f$ at point $a$ is $\lim\limits_{x\to a} \frac{f(x) - f(a)}{x - a}$, \\ denoted by $f'(a)$ provided the limit exists.\\
	$f'(a)=\lim\limits_{x\to a} \frac{f(x) - f(a)}{x - a}=\lim\limits_{h\to0}\frac{f(a+h) - f(a)}{h}=\left.\frac{dy}{dx}\right\rvert_{x=a}$\\
	$f'(a)=$ slope of tangent at pt $a$
	\subsection*{Some properties}
	\begin{descitemize}
	\item $f'(a)$ exists $\Rightarrow f(x)$ is smooth ($\therefore$ continuous) at $a$
	\item $f'(a)$ does not exist at \textbf{discontinuity, corner,} and \textbf{vertical tangent.}
	\end{descitemize}
	Since derivative is limit, if lim from left $\neq$ right, then $f'(a)$ does not exist.
	\subsection*{Formulae}
	\begin{tabular}{c | c}
		\hline
		Function				& Derivative \\ \hline
		$(f(x))^n$			& $nf'(x)f(x)^{n-1}$ \\ \hline
		$\sin f(x)$			& $f'(x)\cos f(x)$ \\ \hline
		$\cos f(x)$			& $-f'(x)\sin f(x)$ \\  \hline
		$\tan f(x)$			& $f'(x)\sec^2 f(x)$ \\ \hline
		$\cot f(x)$			& $-f'(x)\csc^2 f(x)$ \\ \hline
		$\sec f(x)$			& $f'(x)\sec f(x) \tan f(x)$ \\ \hline
		$\csc f(x)$			& $-f'(x)\csc f(x) \cot f(x)$ \\ \hline
		$a^f(x)$				& $f'(x)a^{f(x)} \ln a$ \\ \hline
	\end{tabular}\\
	\begin{tabular}{c | c}
		\hline
		Function				& Derivative \\ \hline
		$k$							& $0$	\\ \hline
		$e^f(x)$				& $f'(x)e^{f(x)}$ \\ \hline
		$\log_af(x)$ 		& $\frac{f'(x)}{f(x) \ln a}$\\ \hline
		$\ln f(x)$			& $\frac{f'(x)}{f(x)}$ \\ \hline
	\end{tabular}
	\begin{tabular}{c | c}
		\hline
		Function				& Derivative \\ \hline
		$\sin^{-1}f(x)$	& $\frac{f'(x)}{\sqrt{1-f(x)^2}}$\\ \hline
		$\cos^{-1}f(x)$	& $-\frac{f'(x)}{\sqrt{1-f(x)^2}}$ \\ \hline
		$\tan^{-1}f(x)$	& $\frac{f'(x)}{1+f(x)^2}$ \\ \hline
	\end{tabular}
	\subsection*{Rules of Differentiation}
	\begin{descitemize}
	\item $(kf)'(x) = kf'(x)$
	\item $(f \pm g)'(x) = f'(x) \pm g'(x)$
	\item $\frac{d}{dx}uv=u\frac{dv}{dx}+v\frac{du}{dx}$
	\item $\left(\frac{f}{g}\right)'(x) = \frac{f'(x)g(x)-f(x)g'(x)}{(g(x))^2}$
	\item $\frac{d}{dx} f(g(x))=f'(g(x)) g'(x)$ or  $\frac{dy}{dx}=\frac{dy}{du}\cdot\frac{du}{dx}$
	\end{descitemize}
	\subsection*{Parametric Differentiation}
		Given $\begin{cases}
		y=u(t)\\
		x=v(t)
		\end{cases}$
		, we have $\frac{dy}{dx}=\frac{\frac{dy}{dt}}{\frac{dx}{dt}}$
		\begin{descitemize}
				\item [Second derivative] \leavevmode \\
				$\frac{d^2y}{dx^2} = \frac{d}{dx}\left(\frac{dy}{dx}\right)$ then do implicit differentiation w.r.t $x$
				\item [Polar equation] ($r=a\theta$): $x=r\cos\theta, y=r\sin\theta$
		\end{descitemize}
		\subsection*{Implicit Differentiation}
		Differentiate w.r.t. to var, then multiply by $\frac{d\text{<var>}}{dx}$
		Common: $y=x^x \iff \ln y = x \ln x$
		\subsection*{Higher Order Derivatives}
		The $n$-th derivative is denoted by $\frac{d^ny}{dx^n}$ or $f^{(n)}(x)$
	\subsection*{Maxima and Minima}
	\begin{descitemize}
		\item [$f(c)$ is Local Maximum] if $f(c) \geq f(x)$ for $x$ near $c$
		\item [$f(c)$ is Local Minimum] if $f(c) \leq f(x)$ for $x$ near $c$
		\item [$f(c)$ is abs maximum] if $f(c) \geq f(x) \forall x \in$ domain
		\item [$f(c)$ is abs minimum] if $f(c) \leq f(x) \forall x \in$ domain
		\item [Critical Point]:\\
		Let $f$ be a function with domain $D$. An interior point (not end-point) $c$ in $D$ is called a \textbf{Critical Point} of $f$ if $f'(c)=0$ of $f'(c)$ does not exist.
		\item [Method to find extreme values of $f$]: \\
					Check critical points of $f$, end-points of domain $D$
		\end{descitemize}
	\subsection*{Method to Find Local Extreme values}
	A function may not have a local extreme at a critical pt. Check using 1st/2nd derivative tests.
	\begin{descitemize}
			\item [1st Derivative Test]: \\
			Assume $c \in (a,b)$ is a critical point of $f$
			\begin{enumerate}
			\item $f'(x)>0$ for $x \in (a,c)$ and $f'(x)<0$ for $x \in (c,b)$, then $f$ is a \textbf{local maximum} 
			\item $f'(x)<0$ for $x \in (a,c)$ and $f'(x)>0$ for $x \in (c,b)$, then $f$ is a \textbf{local minimum} 
			\end{enumerate}
			\item [2nd Derivative Test]: \\
			$f'(c) = 0$ $\begin{cases}
			f''(c)<0 \iff f \text{ has local max at } c \\
			f''(c)>0 \iff f \text{ has local min at } c
			\end{cases}$\\
			Note: if $f'(c)=0$ and $f''(c)=0$ then 2nd derivative test fails. Use 1st derivative test.
	\end{descitemize}
	\subsection*{Method to Find Absolute Extreme Values}
	\begin{enumerate}
	\item Find all critical points $c$ in the interior
	\item Evaluate $f(c)$, where $c$ is a critical or end point
	\item The largest and smallest of these values will be abs max \& min respectively
	\end{enumerate}
	\subsection*{Increasing and Decreasing Functions}
		Test for Monotonic Functions ($f:I$ (interval) $\rightarrow \mathbb{R}$):
		
		\begin{descitemize}
		\item $f'(x)>0$ for any $x$ in $I \Rightarrow f$ is \textbf{increasing} on $I$
		\item $f'(x)<0$ for any $x$ in $I \Rightarrow f$ is \textbf{decreasing} on $I$
		\end{descitemize}
	\subsection*{Concativity}
		$\begin{cases}
		f''(x)<0 \Leftrightarrow f'(x)\text{ is decreasing} \Leftrightarrow \text{Concave Down}\\
		f''(x)>0 \Leftrightarrow f'(x)\text{ is increasing} \Leftrightarrow \text{Concave Up}
		\end{cases}$
	\subsection*{Points of Inflection}
	Let $f:I \rightarrow \mathbb{Z}$ and $c \in I$.\\
	$c$ is a pt of inflection of $f$ if $f$ is continuous at $c$ and the concavity of $f$ changes at $c$.\\
	In another word: c is pt of inflection $\rightarrow f''(c)=0$ (but not the reverse -- c is a pt of inflection only if $f''(c)$ crosses from (+) to (-) and vice versa.)
	
	\section{Integration}
	\subsection*{Indefinite Integral}
	Denoted by $\int f(x)dx = F(x) + C$
	\subsection*{Geometrical Interpretation}
	All curves $y=F(x) + C$ s.t. their slopes at $x$ are $f(x)$
	\subsection*{Rules of Indefinite Integration}
	\begin{enumerate}
	\item $\int kf(x)dx=k\int f(x)dx$
	\item $\int -f(x)dx=-\int f(x)dx$
	\item $\int [f(x) \pm g(x)] dx=\int f(x)dx \pm \int g(x)dx$
	\end{enumerate}
	\subsection*{Integral Formulae}
	\begin{tabular}{c|c}
		\hline
		\textbf{Function}			& \textbf{Integral} \\ \hline
		$\int \cot x dx$			& $\ln (\sin x) + C$ \\ \hline
		$\int \sec x\tan xdx$	& $\sec x+C$\\ \hline
		$\int \csc x\cot xdx$	& $\-\csc x+C$\\ \hline
		$\int \sec^2xdx$			& $\tan x+C$\\ \hline
		$\int \csc^2xdx$			& $-\cot x+C$\\ \hline
		
	\end{tabular}
	
	\begin{tabular}{c|c}
		\hline
		$\int x^n dx$					& $\frac{x^{n+1}}{n+1} +C, n\neq-1, n$ rational\\ \hline
		$\int \frac{1}{\sqrt{a^2-x^2}}dx$ & $\sin^{-1} (\frac{x}{a}) + C$ \\ \hline
		$\int \frac{1}{a^2+x^2}dx$ & $\frac{1}{a}\tan^{-1} (\frac{x}{a}) + C$\\ \hline
		$\int 1 dx = \int dx$	& $x + C$\\ \hline
		$\int e^x dx$					& $e^x + C$ \\ \hline
		$\int a^x dx$					& $\frac{a^x}{\ln a}$ \\ \hline
		$\int \ln x dx$				& $x\ln x - x + C$ \\ \hline
		$\int \frac{1}{x}dx$	& $\ln x + C$ \\ \hline
		$\int \sin kx dx$			& $-\frac{\cos kx}{k} + C$\\ \hline
		$\int \cos kx dx$			& $\frac{\sin kx}{k}+C$\\ \hline
		$\int \tan x dx$			& $\ln(\sec x)+C$ \textbf{or} $-\ln(\sec x)+C$\\ \hline
	\end{tabular}
	
	\begin{tabular}{c|c}
	\hline
	\textbf{Function}			& \textbf{Integral} \\ \hline
	
	
	
	$\int \tan^2 x dx$		& $\tan x - x + C$ \\ \hline
	$\int \sec x dx$			& $\ln(\sec x + \tan x) + C$ \\ \hline
	$\int \csc x dx$			& $\ln(\csc x - \cot x) + C$ \\ \hline
	\end{tabular}
	
	\subsection*{Riemann (Definite) Integrals}
	Riemann sum on $f$ on $[a,b] \approx \sum_{k=1}^{n}f(c_k)\Delta x$\\
	Exact area = $\lim\limits_{n\to\infty}\sum_{k=1}^{n}f(c_k)\Delta x$\\
	\textbf{Riemann Integral of $f$ over $[a,b]$}:\\ $\int_{a}^{b}f(x)dx=\lim\limits_{n\to\infty}\sum_{k=1}^{n}f(c_k)\Delta x$
	\subsection*{Rules of Definite Integrals}
	\begin{enumerate}
	\item $\int_{a}^{a} f(x) dx = 0$, $\int_{a}^{b} kf(x)dx = k\int_{a}^{b}f(x)dx$
	\item $\int_{a}^{b} f(x) dx = -\int_{b}^{a} f(x)dx$
	\item $\int_{a}^{b} [f(x) \pm g(x)] = \int_{a}^{b} f(x) \pm \int_{a}^{b} g(x)$
	\item If $f(x) \geq g(x)$ on $[a,b]$, then $\int_{a}^{b} f(x) dc \geq \int_{a}^{b} g(x) dx$\\
	If $f(x) \geq 0$ on $[a,b]$, then $\int_{a}^{b} f(x)dx \geq 0$
	\item If $f$ is continuous on the interval joining $a,b$ and $c$,\\
	then $ \int_{a}^{b}f(x)dx + \int_{b}^{c}f(x)dx = \int_{a}^{c}f(x)dx $
	\end{enumerate}
	\subsection*{Fundamental Thm of Calculus}
	$F'(x)=f(x)$
	If $F$ is an antiderivative of $f$ on $[a,b]$, then
	$ \int_{a}^{b} F'(x)dx=\int_{a}^{b} f(x)dx=F(b)-F(a) $\\x`
	Let $f$ be continuous on $[a,b]$. Then\\
		$ \frac{d}{dx}\int_{a}^{x}f(t)dt=f(x) $\\
		Note the 2 $x$'s: on $\frac{d}{dx}$ and $\int_{a}^{x}$ and $f(t)$ is indep of $x$
		\begin{enumerate}
		\item $\frac{d}{dx} \int_{0}^{2}t^2dt=0$, $\frac{d}{dx}\int_{0}^{x}\sin \sqrt{t}dt=\sin\sqrt{x}$
		\item $\frac{d}{dx}\left(\int_{1}^{x^4} \frac{t}{\sqrt{t^3+2}} dt\right)=\frac{d}{dx^4}\left(\int_{1}^{x^4} \frac{t}{\sqrt{t^3+2}}dt\right)\frac{dx^4}{dx}\\=\frac{x^4}{\sqrt{(x^4)^3+2}}(4x^3)=\frac{4x^7}{\sqrt{x^12+2}}$
		\item $\frac{d}{dx} \int_{x}^{a} f(t)dt = -\frac{d}{dx}\int_{a}^{x} f(t)dt$
		\item $\frac{d}{dx} \int_{x^2}^{x^4} f(t)dt=\frac{d}{dx} \int_{a}^{x^4}f(t)dt-\frac{d}{dx}\int_{a}^{x^2}f(t)dt$
		\end{enumerate}
	\subsection*{Integration Methods}
	\begin{descitemize}
	\item [Integration by Substitution]:\\
	Use the form $\int f(g(x))dg(x)$ OR use a dummy variable to get to a form in the Integral Formulae (taking into account chain rule)\\
	\begin{tabular}{ c | c | c}
		\hline
		\textbf{Integral}		& \textbf{Sub}	  & \textbf{Use identity} \\ \hline
		$a^2 - u^2$					& $u=a\sin\theta$	& $1-\sin^2\theta=\cos^2\theta$ \\ \hline
		$a^2 + u^2$					& $u=a\tan\theta$	& $1+\tan^2\theta=\sec^2\theta$ \\ \hline
		$u^2-a^2$						& $u=a\sec\theta$	& $sec^2\theta - 1=\tan^2\theta$ \\ \hline
	\end{tabular}
	\item [Integration by Part]:\\
	$ \int uv' dx = uv - \int u'v dx$\\
	Choose $u$ by LIATE (Logarithmic, Inverse trigo, Algebraic, Trigo, Exponential)
	\end{descitemize}
	\subsection*{Area between 2 curves}
	$ A=\int_{a}^{b} (g(x)-f(x)) dx \text{ provided }g(x)\text{ is above }f(x) $
	\subsection*{Volume of a solid}
	$ \text{Volume (around x-axis)} = \int_{a}^{b} \pi y^2 dx $
	
	\section{Series}
	\subsection*{Geometric Series}
	$\sum_{r=1}^{n} ar^{n-1} = a\frac{1-r^n}{1-r}$ \\
	$\sum_{r=1}^{\infty}ar^{n-1} = \frac{a}{1-r}$ if $\left|r\right|<1$, diverges otherwise
	\subsection*{Rules on Series}
	$\sum (a_n \pm b_n) = \sum a_n \pm \sum b_n$, $\sum (ka_n) = k\sum a_n$
	\subsection*{Ratio Test}
	$\lim\limits_{n\to\infty}\left|\frac{a_{n+1}}{a_n}\right| = \rho$, the series $\begin{cases}\text{converges if }&\rho < 1 \\ \text{diverges if }&\rho > 1 \\ \text{no conclusion if }& \rho=1\end{cases}$
	\subsection*{p-series}
	$\sum\limits_{n=1}^{\infty} \frac{1}{n^p} \begin{cases}\text{diverges}&0\leq p\leq1 \\ \text{converges }&p>1 \end{cases}$
	\subsection*{Radius of convergence ($R$)}
	Use the \textbf{Ratio Test} to find \textbf{range} of convergence of \textbf{Power Series} about $x=a$, $\sum_{n=0}^{\infty} c_n(x - a)^n$
	\begin{enumerate}
	\item $R=0$, converges only at $a$
	\item $R=h$, converges in $(a-h, a+h)$ but diverges outside
	\item $R=\infty$, converges at every $x$ 
	\end{enumerate}
	\subsection*{Differentiation and Integration of Power Series}
	Let $f(x) = \sum_{n=0}^{\infty} c_n(x-a)^n, a-h<x<a+h$ where $h$ is Radius of Convergence, then for $a-h<x<a+h$, \\ $f'(x)=\sum_{n=0}^{\infty}\frac{d}{dx}(c_n(x-a)^n)=\sum_{n=1}^{\infty}nc_n(x-a)^{n-1}$\\
	$f''(x)=\sum\limits_{n=1}^{\infty} nc_n\frac{d}{dx}(x-a)^{n-1}=\sum\limits_{n=2}^{\infty} n(n-1)c_n(x-a)^{n-2}$\\
	Note lower bound of sum increases by 1\\
	$\int^x_0 f(x)dx=\int^x_0\sum\limits_{n=0}^{\infty}c_n(x-a)^n=\sum\limits_{n=0}^{\infty}c_n\frac{(x-a)^{n+1}}{n+1}$\\
	The radius of convergence is $h$ after diff and integ
	\subsection*{Taylor Series of $f$ at $a$}
	$f(x) = \sum_{k=0}^{\infty}\frac{f^{(k)}(a)}{k!}(x-a)^k$
	\subsection*{MacLaurin Series}
	Taylor series of $f$ at $0$, i.e.	$f(x) = \sum_{n=0}^{\infty} \frac{f^{(n)}(0)}{n!}x^n$
	\subsection*{List of common MacLaurin Series}
	\begin{enumerate}
	\item $\frac{1}{1-x} = \sum_{n=0}^{\infty} x^n, -1<x<1, R=1$
	\item $\frac{1}{1+x} = \sum_{n=0}^{\infty} (-1)^n x^n, -1<x<1, R = 1$
	\item $\frac{1}{1+x^2} = \sum_{n=0}^{\infty} x^2n, -1<x<1, R=1$
	\item $ln(1+x) = \sum_{n=1}^{\infty} \frac{(-1)^{n-1}x^n}{n}, -1<x<1, R=1$
	\item $\sin x = \sum_{n=0}^{\infty}\frac{(-1)^n x^{2n+1}}{(2n+1)!}, -\infty<x<\infty, R=\infty$
	\item $\cos x = \sum_{n=1}^{\infty}\frac{(-1)^n x^2n}{(2n)!}, -\infty<x<\infty, R=\infty$
	\item $e^x = \sum_{n=0}^{\infty} \frac{x^n}{n!}, -\infty<x<\infty, R=\infty$
	\item $\tan^{-1} x = \sum_{n=0}^{\infty} \frac{(-1)^n}{2n+1}x^{2n+1}, -1\leq x\leq 1, R=1$
	\item $\frac{1}{(1-x)^2}=\sum_{n=1}^{\infty}nx^{n-1},-1<x<1, R = 1$
	\item $\frac{1}{(1-x)^3}=\frac{1}{2}\sum_{n=2}^{\infty}n(n-1)x^{n-2},-1<x<1, R = 1$
	\item $(1+x)^k=\sum_{n=0}^{\infty} {k \choose n}x^n, -1<x<1,R = 1$
	\item $(1+x)^n = 1 + nx + \frac{n(n-1)}{2!}x^2 + \frac{n(n-1)(n-2)}{3!}x^3+..., -1<x<1, R=1$
	\end{enumerate}
	\subsection*{Taylor Polynomials}
	The $n$-th order Taylor Polynomial of $f$ at $a$\\
	$P_n(x) = \sum_{k=0}^{n}\frac{f^{(k)}(a)}{k!}(x-a)^k$\\
	It gives a good polynomial approxn of order $n$
	\subsection*{Taylor's Theorem}
	$f(x) = P_n(x) + R_n(x)$ where\\
	$R_n(x)=\frac{f^{(n+1)}(c)}{(n+1)!}(x-a)^{n+1}$ for $a<c<x$.\\
	$R_n(x)$ is \textbf{remainder of order $n$} or \textbf{error term}
	
	\section{Vector}
	\subsection*{Dot Product}
	$v_1 = \begin{pmatrix}x_1\\y_1\\z_1\end{pmatrix}, v_2 = \begin{pmatrix}x_2\\y_2\\z_2\end{pmatrix}$, $v_1 \cdot v_2 = x_1x_2 + y_1y_2 + z_1z_2$
	
	$\cos\theta = \frac{v_1\cdot v_2}{\lVert v_1 \rVert \  \lVert v_2 \rVert}$, Projection of $b$ onto $a = \frac{b\cdot a}{\rVert a \lVert^2}a$
	
	Commut, assoc, distr, and $v_1 \cdot v_1 = \|v_1\|^2$
	\subsection*{Cross Product}
	$v_1 \times v_2 = (y_1z_2 - y_2z_1)\mathbf{i} - (x_1z_2-x_2z_1)\mathbf{j} + (x_1y_2 - x_2y_1)\mathbf{k}$
	Area of parallellogram = $\lVert v_1 \times v_2 \rVert = \| v_1 \| \ \| v_2 \|\ \sin\theta$
	
	Distr, assoc, but $v_1 \times v_2 = -v_2 \times v_1$ and $v_1 \times v_1 = O$
	\section{Functions of Several Variables}
	\subsection*{Partial Derivatives}
	of $z=f(x,y)$ w.r.t. $x$ is denoted by $\left. \frac{\partial z}{\partial x} \right|_{(a,b)}$ or $f_x(a,b)$
	
	Method: Fix the other variable (Note: $f_{xy} = f_{yx}$)
	\subsection*{Chain Rule}
	$\frac{dz}{dt} = \frac{\partial z}{\partial x} \cdot \frac{dx}{dt} + \frac{\partial z}{\partial y}\cdot\frac{dy}{dt}$ AND $\frac{dw}{dt} = \frac{\partial w}{\partial x} \cdot \frac{dx}{dt} + \frac{\partial w}{\partial y}\cdot\frac{dy}{dt} + \frac{\partial w}{\partial z}\cdot\frac{dz}{dt}$
	\begin{wrapfigure}[6]{l}{0pt}
	\raisebox{0pt}[\dimexpr\height-0.6\baselineskip\relax]{
	\begin{tikzpicture}
		\draw [->] (0,0) -- (0.75,0.5) node [midway, above, scale = 0.7] {$\frac{\partial z}{\partial x}$};
		\draw [->] (0,0) -- (0.75,-0.5) node [midway, below, scale = 0.7] {$\frac{\partial z}{\partial y}$};
		\draw [->] (0.85,0.5) -- (1.6,0.05) node [midway, above, scale = 0.7] {$\frac{dx}{dt}$};
		\draw [->] (0.85,-0.5) -- (1.6,-0.05) node [midway, below, scale = 0.7] {$\frac{dy}{dt}$};
		\draw [fill, red] (0,0) circle [radius=0.05];
		\draw [fill, red] (0.8,0.5) circle [radius=0.05];
		\draw [fill, red] (0.8,-0.5) circle [radius=0.05];
		\draw [fill, red] (1.6,0) circle [radius=0.05];
		\node [below] at (0,0) {$z$};
		\node [below] at (0.8,0.5) {$x$};
		\node [below] at (0.8,-0.5) {$y$};
		\node [below] at (1.6,0) {$t$};
		\end{tikzpicture}
		\begin{tikzpicture}
			\draw [->] (0,0) -- (1.2,0.5) node [midway, above, scale = 0.7] {$\frac{\partial w}{\partial x}$};
			\draw [->] (0,0) -- (1.2,-0.5) node [midway, below, scale = 0.7] {$\frac{\partial w}{\partial z}$};
			\draw [->] (0,0) -- (1.2,0) node [midway, scale = 0.6] {$\frac{\partial w}{\partial y}$};
			\draw [->] (1.3,0.5) -- (2.5,0.05) node [midway, above, scale = 0.7] {$\frac{dx}{dt}$};
			\draw [->] (1.3,-0.5) -- (2.5,-0.05) node [midway, below, scale = 0.7] {$\frac{dz}{dt}$};
			\draw [->] (1.3,0) -- (2.45,0) node [midway, scale = 0.7] {$\frac{dy}{dt}$};
			\draw [fill, red] (0,0) circle [radius=0.05];
			\draw [fill, red] (1.25,0.5) circle [radius=0.05];
			\draw [fill, red] (1.25,-0.5) circle [radius=0.05];
			\draw [fill, red] (1.25,0) circle [radius=0.05];
			\draw [fill, red] (2.5,0) circle [radius=0.05];
			\node [below] at (0,0) {$w$};
			\node [below] at (1.2,0.5) {$x$};
			\node [below] at (1.2,-0.5) {$z$};
			\node [below] at (1.2,0) {$y$};
			\node [below] at (2.5,0) {$t$};
		\end{tikzpicture}
	}
		\end{wrapfigure}
		
		$z = f(x,y), \\x=x(t), y=y(t)$
		
		$w = f(x,y,z), \\x=x(t), \\y=y(t), z=z(t)$
		
		\subsection*{Directional Derivative}
		$f_x(a,b)$ is rate of change of $f$ along direction of x-axis
		
		Directional derivative of $f$ at $(a,b)$ in direction of unit vector $u = u_1\mathbf{i} + u_2\mathbf{j}$ is $D_uf(a,b)=f_x(a,b)u_1 + f_y(a,b)u_2$
		
		or $D_uf(a,b,c)=f_x(a,b,c)u_1 + f_y(a,b,c)u_2 + f_z(a,b,c)u_3$
		
		 $df = D_uf(a,b)\cdot dt$ (normal $\cdot$ multiplication) measures change in $f$ ($df$) when we move a small distance $dt$, and $u$ is the unit directional vector of the change and $(a,b)$ is the original pt 
		 
		 \subsection*{Gradient Vector}
		 Denoted by $\nabla f = f_x\mathbf{i} + f_y\mathbf{j}$ where
		 
		 $\nabla f(a,b) \cdot u = D_uf(a,b) = \lVert\nabla f(a,b)\rVert\cos\theta$
		 		 
		 $D_uf(a,b) >0$ and max when $\cos \theta = 1 \iff \theta = 0^{\circ}$
		 
		 $D_uf(a,b) <0$ and min when $\cos \theta = -1 \iff \theta = 180^{\circ}$
		 
		 \subsection*{Max and Min Values}
		 \textbf{Critical Points - First Derivative Test}
		 
		 $f$ has a local max or min at $(a,b) \land f_x$ exists $\land f_y$ exists $\rightarrow$ $f_x=0 \land f_y=0$ (But not the converse)
		 
		\textbf{Second Derivative Test}
		
		Disriminant = $f_{xx}(a,b)f_{yy}(a,b) - f_{xy}(a,b)^2$
		\begin{enumerate}
		\item $D>0 \land f_{xx}(a,b)>0 \rightarrow f$ has	a local min at $(a,b)$
		\item $D>0 \land f_{xx}(a,b)<0 \rightarrow f$ has a local max at $(a,b)$
		\item $D<0 \rightarrow f$ has a saddle-point at $(a,b)$
		\item $D = 0 \rightarrow$ no conclusion
		\end{enumerate}
		 
		 
		 \section{Ordinary Differential Equation (ODE)}
		 \textbf{No Crossing Principle}: solution curves do not cross each other
		 \subsection*{First order ODE}
		 There is only 1 soln for initial value problem with 1st order ODE. Intersection point of 2 curves is the initial pt.
		 \begin{enumerate}
		 \item $\frac{dy}{dx} = \frac{M(x)}{N(y)} \iff \int M(x)dx = \int N(y)dy$
		 \item $y' = f(\frac{y}{x}) \Leftrightarrow $ Let $v=\frac{y}{x}, f(v) = y'\Leftrightarrow \frac{dv}{f(v)-v}=\frac{dx}{x}$
		 \item $y'=\frac{ax+by+c}{a_1x+b_1y+c_1} \Leftrightarrow$ Let $u = ax+by$
		 \item $\frac{dy}{dx}+p(x)y = Q(x) \Leftrightarrow ye^{\int p(x)dx}=\int Q(x)e^{\int p(x)dx}dx$
		 \item \textbf{Bernoulli eqn} $y' + p(x)y = q(x)y^n \Leftrightarrow $\\
		 Let $z=y^{1-n} \Leftrightarrow z' + (1-n)p(x)z = (1-n)q(x)$  
		 \end{enumerate}
		 \subsection*{Scenarios}
		 Radioactive decay: $\frac{dx}{dt} = kx$, $x(t)=x(0)e^{-\frac{\ln 2}{\tau}t}$
		 
		 Uranium-Thorium: ${\frac{T}{U}=\frac{k_U}{k_T-k_U}(1-e^{-(k_T-k_U)t}) k_N = \frac{\ln 2}{\tau_N}}$
		 
		 Cooling/Heating: $\int\frac{dT}{T-T_0}=\int kdt$, $T(t) - T_0 = (T(0) - T_0)e^{kt}$
		 
		 Retarded fall: $m\frac{dv}{dt} = mg-bv^2,$\\
		 $v = k\frac{1 + ce^{-pt}}{1-ce^{-pt}}, k^2 = \frac{mg}{b},c = \frac{v(0)-k}{v(0)+k}, p=\frac{2kb}{m}$
		 \subsection*{Hyperbolic Functions}
		 ${\cosh x=\frac{e^x + e^{-x}}{2}, \sinh x = \frac{e^x + e^{-x}}{2}, \tanh x=\frac{\sinh x}{\cosh x}=\frac{1-e^{-2x}}{1+e^{-2x}}}$
		 
		 \subsection*{Second order linear ODE}
		 Form: $\frac{d^2y}{dx^2}+p(x)\frac{dy}{dx}+q(x)y=F(x)$
		 
		 homogeneous $\Leftrightarrow F(x)=0$, else non-homogeneous
		 
		 
		 
		\subsection*{Homogeneous 2nd order linear ODE}
		\textbf{Linearly dependent} ${\Leftrightarrow \forall x \exists c\text{ s.t. }u(x)=cv(x)}$
		
		$y_1, y_2$ are lin. indep. solns $\Rightarrow$ a general soln is $y=c_1y_1+c_2y_2$
		
		$y_1,y_2$ are NOT lin. indep. solns $\Rightarrow y=c_1y_1+c_2y_2$ is a soln but not a general soln
		
		$\frac{d^2y}{dx^2}+A\frac{dy}{dx}+By=0$ has the trivial soln $y=0$ and non-trivial soln: Let $y=e^{\lambda x}$, solve $\lambda^2+A\lambda+B=0$, general soln is: (PS: Reverse is $A=-(\lambda_1 + \lambda_2), B=\lambda_1 \lambda_2$)
		\begin{enumerate}
		\item 2 real roots: ${y=c_1e^{\lambda_1x}+c_2e^{\lambda_2x}}$
		\item 1 real root: $y=c_1e^{\lambda_1x}+c_2xe^{\lambda_1x}$
		\item 2 complex roots ($a+ib$): ${y=e^{ax}(c_1\cos bx + c_2\sin bx)}$
		\end{enumerate}
		\section{Mathematical Modelling (B = birth rate, D = death rate)}
		\subsection*{Malthusian Population Growth }
		$N(t) = N(0)e^{kt}$ where $k=B-D$. Conditions:
		\begin{enumerate}
		\item $k>0$ ($B>D$): popn explosion ($e^{kt}\to\infty, N(t)\to\infty$ as $t\to\infty$)
		\item $k=0$ ($B=D$): stable ($N(t)=N(0)$ for all $t$)
		\item $k<0$ ($B<D$): extinction ($e^{kt}\to0, N(t)\to0$ as $t\to\infty$)
		\end{enumerate}
		\subsection*{Logistic Growth Model}
		\begin{tikzpicture}
					\begin{axis}[
						xmin=-6,xmax=6,xticklabels={,,},
						axis x line=bottom,
						ytick={0,.5,1},yticklabels={,,},ymax=1,
						axis y line=left,
						height=3.5cm
					]
						\addplot[blue,mark=none,samples=100,domain=-6:6](x,{1/(1+exp(-x))});
						\addplot[blue,mark=none,samples=100,domain=-6:6](x,{1/(1+exp(-x))});
					\end{axis}
				\end{tikzpicture}\\
		Eqn: $\frac{dN}{dt}=(B-D)N, N(0) = \hat{N}, N_\infty=\frac{B}{s}$\\
			$\frac{dN}{dt}=(B-D)N = (B-sN)N = BN-sN^2$ where $s$ is a small number compared to $B$.\\
		$\frac{dN}{dt}=0$ when $N \approx \frac{B}{s}$ (population stops growing)\\
				This constant $\frac{B}{s}$ is called carrying capacity, sustainable population, or logistic equilibrium population. Or that the population stabilises at $\frac{B}{s}$\\
		$N(t) = \frac{B}{s+\left(\frac{B}{N_0}-s\right)e^{-Bt}}= \frac{N_\infty}{1 + (\frac{N_\infty}{N_0}-1)e^{-Bt}}$\\
		$\lim\limits_{t\to\infty} N(t) = \frac{B}{s}$\\
		\textbf{Case 1: $B - sN(t) > 0\ \forall t$ (Popn < sustainable popn)}\\
		Logistic curve increasing\\
		\textbf{Case 2: $B -sN(t) < 0$ at all $t$ (Popn > sustainable popn)}\\
		Logistic curve decreasing\\
		\textbf{Case 3: $B -sN(t) = 0$ at all $t$ (At sustainable popn)}\\
		Population constant at $N(0)$\\
		\begin{tikzpicture}
							\begin{axis}[
								xmajorticks=false,
								xmin=0,xmax=4,
								axis x line=bottom,
								ymajorticks=false,
								ymin=0,ymax=2,
								axis y line=left,
								height=3.5cm,
								clip=false
							]
								\addplot[blue,mark=none,samples=100,domain=0:4](x,{1/(1+exp(-x))}) node[below,pos=1] {Case 1};
								\addplot[red,mark=none,samples=100,domain=0:4](x,{2-1/(1+exp(-x))})node[above,pos=1] {Case 2};
								\addplot[brown,mark=none,samples=100,domain=0:4](x,{1})node[right,pos=1] {Case 3};
							\end{axis}
			\end{tikzpicture}\\
			\subsection*{Harvesting}
			$\frac{dN}{dt}=BN-sN^2-E$ where $E$ is fish caught per year.\\
			\textbf{DO NOT ATTEMPT TO SOLVE THE ODE.} They will just ask to draw graph.\\
			Method:
			\begin{enumerate}
			\item Let $F(N) = \frac{dN}{dt} = -sN^2 + BN -E$
			\item Discriminant = $B^2 - 4(-s)(-E) = B^2 - 4sE$
			\item Cases:
			\begin{enumerate}
			\item $D<0$: No equiblirium soln (Popn is decreasing to extinction)\\
			Note: $-s<0$, shape is $\cap$, $F(N)\neq0$\\
			\begin{tikzpicture}
										\begin{axis}[
											xmajorticks=false,
											xmin=-2,xmax=6,
											xlabel={$N$},
											axis x line=center,
											ymajorticks=false,
											ymin=-10,ymax=2,
											ylabel={$F(N)$},
											axis y line=center,
											height=3cm,
										]
										\addplot[blue,mark=none,samples=100,domain=-2:6](x,{-2-(x-2)^2});
										\end{axis}
						\end{tikzpicture}
						\begin{tikzpicture}
																\begin{axis}[
																	xmajorticks=false,
																	xmin=0,xmax=2,
																	xlabel={$t$},
																	axis x line=center,
																	ymajorticks=false,
																	ymin=0,ymax=7,
																	ylabel={$N$},
																	axis y line=center,
																	height=3cm,
																]
																\addplot[blue,mark=none,samples=100,domain=-2:6](x,{6-1/3*x^3 + 2*x^2 - 6*x});
																\end{axis}
												\end{tikzpicture}
			\item $D>0$: 2 equilibrium solns\\
			Solve $F(N)$ for $\beta_1, \beta_2$ where $\beta_1<\beta_2<\frac{B}{s}$\\
			There are 3 possible cases:\\
				\begin{tikzpicture}
																			\begin{axis}[
																				xmajorticks=false,
																				xmin=0,xmax=1.8,
																				xlabel={$t$},
																				axis x line=center,
																				ymajorticks=false,
																				ymin=0,ymax=25,
																				ylabel={$N$},
																				axis y line=center,
																				height=4cm,
																				clip=false
																			]
																			\addplot[blue,mark=none,samples=100,domain=0:1.6](x,{24-1/3*x^3 + 2*x^2 - 6*x}) node[above,pos=1] {Case 1};
																			\addplot[brown,mark=none,samples=100,domain=0:1.6](x,{12-(-1/3*x^3 + 2*x^2 - 6*x)}) node[below,pos=1] {Case 2};
																			\addplot[red,mark=none,samples=100,domain=0:1.6](x,{6-1/3*x^3 + 2*x^2 - 6*x}) node[below,pos=1] {Case 3};
																			\addplot[black,mark=none,samples=100,domain=0:1.7](x, {18}) node [right, pos=1] {$\beta_2$};
																			\addplot[black,mark=none,samples=100,domain=0:1.7](x, {6}) node [right, pos=1] {$\beta_1$};
																			\end{axis}
															\end{tikzpicture}\\
			$\frac{B}{s}=\beta_1+\beta_2, \frac{E}{s}=\beta_1\beta_2$\\
			$\beta_2$ is stable ($N(0)$ slightly diff from $\beta_2$, popn will still tend to $\beta_2$). $\beta_1$ is not stable ($N(0)$ slightly diff from $\beta_1$ will not tend to $\beta_1$)
			\item $D=0$: 1 equilibrium solns\\
			\begin{tikzpicture}
							\begin{axis}[
								xmajorticks=false,
								xmin=0,xmax=1.8,
								xlabel={$t$},
								axis x line=center,
								ymajorticks=false,
								ymin=0,ymax=13,
								ylabel={$N$},
								axis y line=center,
								height=3cm,
								clip=false
							]
							\addplot[blue,mark=none,samples=100,domain=0:1.6](x,{12-1/3*x^3 + 2*x^2 - 6*x}) node[above,pos=1] {Case 1};
							\addplot[red,mark=none,samples=100,domain=0:1.6](x,{6-1/3*x^3 + 2*x^2 - 6*x}) node[below,pos=1] {Case 2};
							\addplot[black,mark=none,samples=100,domain=0:1.7](x, {6}) node [right, pos=1] {$\frac{B}{2s}$};
							\end{axis}
				\end{tikzpicture}\\
			Suppose $N(0)>\frac{B}{2s}$ then max. harvesting w/o extinction $E=\frac{B^2}{4s}$
			\end{enumerate}
			\end{enumerate}
		PS: more precise curves, follow the original logistic growth model graph (S-shaped) increasing: gentle-steep-gentle, decreasing: steep-gentle-steep
		\subsection*{Additional Notes}
		\begin{descitemize}
		\item If the question is in powers above 2, e.g. $\frac{dN}{dt}=aN^4+bN^3+cN^2+dN+e$, the same rule about the graph still applies: the stable populations are the solutions to $aN^4+bN^3+cN^2+dN+e=0$
		\item If there is no harvesting, then $N=0$ is also a solution.
		\end{descitemize}
		
	
\end{multicols*}
\end{document} 