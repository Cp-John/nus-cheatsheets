\documentclass[10pt,landscape,a4paper]{article}
%\usepackage[utf8]{inputenc}
%\usepackage[ngerman]{babel}
\usepackage[normalem]{ulem}
\usepackage{tikz}
\usetikzlibrary{shapes,positioning,arrows,fit,calc,graphs,graphs.standard}
\usepackage[nosf]{kpfonts}
\usepackage[t1]{sourcesanspro}
%\usepackage[lf]{MyriadPro}
%\usepackage[lf,minionint]{MinionPro}
\usepackage{multicol}
\usepackage{wrapfig}
\usepackage[top=0mm,bottom=1mm,left=0mm,right=1mm]{geometry}
\usepackage[framemethod=tikz]{mdframed}
\usepackage{microtype}
%\usepackage{physics}
\usepackage{tabularx}
\usepackage{hhline}
\usepackage{makecell}
\usepackage{mathtools}

\usepackage{listings}

\DeclarePairedDelimiter{\ceil}{\lceil}{\rceil}

\newcommand\codeblue[1]{\textcolor{blue}{\code{#1}}}

\usepackage{lastpage}
\usepackage{datetime}
\yyyymmdddate
\renewcommand{\dateseparator}{-}
\let\bar\overline

\definecolor{myblue}{cmyk}{1,.72,0,.38}

\def\firstcircle{(0,0) circle (1.5cm)}
\def\secondcircle{(0:2cm) circle (1.5cm)}

\colorlet{circle edge}{myblue}
\colorlet{circle area}{myblue!5}

\tikzset{filled/.style={fill=circle area, draw=circle edge, thick},
  outline/.style={draw=circle edge, thick}}

\pgfdeclarelayer{background}
\pgfsetlayers{background,main}

%\everymath\expandafter{\the\everymath \color{myblue}}
%\everydisplay\expandafter{\the\everydisplay \color{myblue}}


\renewcommand{\baselinestretch}{.8}
\pagestyle{empty}

\global\mdfdefinestyle{header}{%
  linecolor=gray,linewidth=1pt,%
  leftmargin=0mm,rightmargin=0mm,skipbelow=0mm,skipabove=0mm,
}

\newcommand{\header}{
  \begin{mdframed}[style=header]
    \footnotesize
    \sffamily
    CS2105 Finals Cheatsheet v1.0 (\today)\\
    by~Julius Putra Tanu Setiaji,~page~\thepage~of~\pageref{LastPage}
  \end{mdframed}
}

\let\counterwithout\relax
\let\counterwithin\relax
\usepackage{chngcntr}

\usepackage{verbatim}

\usepackage{etoolbox}
\makeatletter
\preto{\@verbatim}{\topsep=0pt \partopsep=0pt }
\makeatother

\counterwithin*{equation}{section}
\counterwithin*{equation}{subsection}
\usepackage{enumitem}
\newlist{legal}{enumerate}{10}
\setlist[legal]{label*=\arabic*.,leftmargin=2.5mm}
\setlist[itemize]{leftmargin=3mm}
\setlist[enumerate]{leftmargin=3.5mm}
\setlist{nosep}
\usepackage{minted}

\def\code#1{\texttt{#1}}

\newenvironment{descitemize} % a mixture of description and itemize
{\begin{description}[leftmargin=*,before=\let\makelabel\descitemlabel]}
    {\end{description}}

\newcommand{\descitemlabel}[1]{%
  \textbullet\ \textbf{#1}%
}
\makeatletter



\renewcommand{\section}{\@startsection{section}{1}{0mm}%
  {.2ex}%
  {.2ex}%x
  {\color{myblue}\sffamily\small\bfseries}}
\renewcommand{\subsection}{\@startsection{subsection}{1}{0mm}%
  {.2ex}%
  {.2ex}%x
  {\sffamily\bfseries}}
\renewcommand{\subsubsection}{\@startsection{subsubsection}{1}{0mm}%
  {.2ex}%
  {.2ex}%x
  {\rmfamily\bfseries}}



\def\multi@column@out{%
  \ifnum\outputpenalty <-\@M
  \speci@ls \else
  \ifvoid\colbreak@box\else
    \mult@info\@ne{Re-adding forced
      break(s) for splitting}%
    \setbox\@cclv\vbox{%
      \unvbox\colbreak@box
      \penalty-\@Mv\unvbox\@cclv}%
  \fi
  \splittopskip\topskip
  \splitmaxdepth\maxdepth
  \dimen@\@colroom
  \divide\skip\footins\col@number
  \ifvoid\footins \else
    \leave@mult@footins
  \fi
  \let\ifshr@kingsaved\ifshr@king
  \ifvbox \@kludgeins
    \advance \dimen@ -\ht\@kludgeins
    \ifdim \wd\@kludgeins>\z@
      \shr@nkingtrue
    \fi
  \fi
  \process@cols\mult@gfirstbox{%
    %%%%% START CHANGE
    \ifnum\count@=\numexpr\mult@rightbox+2\relax
      \setbox\count@\vsplit\@cclv to \dimexpr \dimen@-1cm\relax
      \setbox\count@\vbox to \dimen@{\vbox to 1cm{\header}\unvbox\count@\vss}%
    \else
      \setbox\count@\vsplit\@cclv to \dimen@
    \fi
    %%%%% END CHANGE
    \set@keptmarks
    \setbox\count@
    \vbox to\dimen@
    {\unvbox\count@
      \remove@discardable@items
      \ifshr@nking\vfill\fi}%
  }%
  \setbox\mult@rightbox
  \vsplit\@cclv to\dimen@
  \set@keptmarks
  \setbox\mult@rightbox\vbox to\dimen@
  {\unvbox\mult@rightbox
    \remove@discardable@items
    \ifshr@nking\vfill\fi}%
  \let\ifshr@king\ifshr@kingsaved
  \ifvoid\@cclv \else
    \unvbox\@cclv
    \ifnum\outputpenalty=\@M
    \else
      \penalty\outputpenalty
    \fi
    \ifvoid\footins\else
      \PackageWarning{multicol}%
      {I moved some lines to
        the next page.\MessageBreak
        Footnotes on page
        \thepage\space might be wrong}%
    \fi
    \ifnum \c@tracingmulticols>\thr@@
      \hrule\allowbreak \fi
  \fi
  \ifx\@empty\kept@firstmark
    \let\firstmark\kept@topmark
    \let\botmark\kept@topmark
  \else
    \let\firstmark\kept@firstmark
    \let\botmark\kept@botmark
  \fi
  \let\topmark\kept@topmark
  \mult@info\tw@
  {Use kept top mark:\MessageBreak
    \meaning\kept@topmark
    \MessageBreak
    Use kept first mark:\MessageBreak
    \meaning\kept@firstmark
    \MessageBreak
    Use kept bot mark:\MessageBreak
    \meaning\kept@botmark
    \MessageBreak
    Produce first mark:\MessageBreak
    \meaning\firstmark
    \MessageBreak
    Produce bot mark:\MessageBreak
    \meaning\botmark
    \@gobbletwo}%
  \setbox\@cclv\vbox{\unvbox\partial@page
    \page@sofar}%
  \@makecol\@outputpage
  \global\let\kept@topmark\botmark
  \global\let\kept@firstmark\@empty
  \global\let\kept@botmark\@empty
  \mult@info\tw@
  {(Re)Init top mark:\MessageBreak
    \meaning\kept@topmark
    \@gobbletwo}%
  \global\@colroom\@colht
  \global \@mparbottom \z@
  \process@deferreds
  \@whilesw\if@fcolmade\fi{\@outputpage
      \global\@colroom\@colht
      \process@deferreds}%
    \mult@info\@ne
    {Colroom:\MessageBreak
      \the\@colht\space
      after float space removed
      = \the\@colroom \@gobble}%
    \set@mult@vsize \global
  \fi}
\global\let\tikz@ensure@dollar@catcode=\relax

\def\mathcolor#1#{\@mathcolor{#1}}
\def\@mathcolor#1#2#3{%
  \protect\leavevmode
  \begingroup
  \color#1{#2}#3%
  \endgroup
}

\makeatother
\setlength{\parindent}{0pt}

\setminted{tabsize=2, breaklines}
% Remove belowskip of minted
\setlength\partopsep{-\topsep}

\setlength\columnsep{1.5pt}
\setlength\columnseprule{0.1pt}

\begin{document}
\setlength{\abovedisplayskip}{0pt}
\setlength{\belowdisplayskip}{0pt}


\scriptsize
\begin{multicols*}{4}
  \raggedcolumns
  \section{Computer Networks and the Internet}
  \subsection{What is the Internet}
  \subsection{Network Edge}
  \subsection{Network Core}
  \subsubsection{Circuit switching}
  Dedicated circuit per call
  \begin{itemize}
    \item call setup required
    \item circuit-like (guaranteed) performance
    \item circuit segment idle if not used (no sharing)
  \end{itemize}
  \subsubsection{Packet Switching}
  Data sent through the net in discrete chunks
  \begin{itemize}
    \item \textbf{Store-and-forward}: entire packet must arrive at a router before it can be transmitted to the next link.
    \item \textbf{Addressing}: each packet needs to carry source and destination information
    \item Users share network resources
    \item Resources are used on demand
    \item Excessive congestion is possible
  \end{itemize}

  \subsection{Delay, Loss and Throughput in Networks}
  End-to-end packet delay consisting of 4 sources
  \subsubsection{4 Sources of Packet Delay}
  \begin{itemize}
    \item Nodal Processing ($d_{proc}$): check bit errors, determine output link, typically $<$ msec
    \item Queueing ($d_{queue}$): waiting in queue for transmission, depends on congestion level of router
    \item Transmission ($d_{trans} = \frac{L}{R}$): L = packet length (bits), R = link bandwidth (bps)
    \item Propagation ($d_{prop} = \frac{d}{s}$): d = length of physical link, s = propagation speed in medium ($2\times 10^8m/sec$)
  \end{itemize}
  \subsubsection{Throughput}
  \begin{itemize}
    \item How many bits can be transmitted per unit time
    \item Measured for end-to-end communication. Compare with link capacity (bandwidth) only for specific link
  \end{itemize}
  \subsubsection{Units}
  \begin{itemize}
    \item 1 byte = 8 bits
    \item (-) Prefixes: milli, micro, nano, pico, femto, atto, zepto, yocto
    \item (+) Prefixes: kilo, mega, giga, tera, peta, exa, zetta, yotta
  \end{itemize}

  \subsection{Protocol Layers and Service Models}
  \subsubsection{5 Layers}
  \begin{itemize}
    \item \textbf{Application}: supporting network applications, e.g. FTP, SMTP, HTTP
    \item \textbf{Transport}: process-to-process data transfer, e.g. TCP, UDP
    \item \textbf{Network}: routing of datagrams from source to destination, e.g. IP, routing protocols
    \item \textbf{Link}: Data transfer between neighbouring network elements, e.g. Ethernet, 802.11, PPP
    \item \textbf{Physical}: ``on the wire''
  \end{itemize}

  \subsubsection{ISO/OSI Reference Model}
  Theoretical only, 2 additional layers between \textbf{Application} and \textbf{Transport}: \textbf{Presentation} (allow applications to interpret meaning of data, e.g. encryption, compression, machine-specific convention) and \textbf{Session} (synchronisation, checkpointing, recovery of data exchange)

  \section{Application Layer}
  \subsection{Principles of Network Applications}
  \subsubsection{Client-Server}
  \begin{itemize}
    \item \textbf{Server}: waits for incoming requests, provides requested service to client, data centers for scaling
    \item \textbf{Client}: initiates contact with server, typically requests service from server, For web, client is usually implemented in browser
  \end{itemize}
  \subsubsection{Peer-to-Peer (P2P)}
  \begin{itemize}
    \item No always-on server
    \item Arbitrary end systems directly communicate.
    \item Peers request service from other peers, provide service in return to other peers
    \item \textbf{Self-scalability}: new peers bring new service capacity, as well as new service demands
    \item Peers are intermittently connected and change IP addresses (complex management)
  \end{itemize}
  \subsubsection{Requirements of apps}
  \begin{itemize}
    \item \textbf{Data integrity}: 100\% reliable vs some data loss
    \item \textbf{Timing}: some apps require low delay to be ``effective''
    \item \textbf{Throughput}
    \item \textbf{Security}
  \end{itemize}
  \subsubsection{Definition of App-layer Protocols}
  \begin{itemize}
    \item \textbf{Types of Messages exchanged,} e.g. request, response
    \item \textbf{Message syntax}, e.g. message fields and how they are delineated
    \item \textbf{Message semantics}: meaning of information in fields
    \item \textbf{Rules} for when and how application send and respond to messages
  \end{itemize}
  \subsubsection{Transport-Layer Protocols}
  \begin{tabularx}{\columnwidth}{|X|X|}
    \hline
    \textbf{TCP}                                                              & \textbf{UDP}                                                      \\
    \hline
    \textbf{Reliable} data transfer                                           & \textbf{Unreliable} data transfer                                 \\
    \hline
    \textbf{Flow control}: sender won't overwhelm receiver                    & \textbf{No flow control}                                          \\
    \hline
    \textbf{Congestion control}: throttle sender when network is overloaded   & \textbf{No congestion control}                                    \\
    \hline
    \textbf{Does not provide}: timing, minimum throughput guarantee, security & \textbf{Does not provide}: timing, throughput guarantee, security \\
    \hline
  \end{tabularx}

  \subsection{Web and HTTP}
  \subsubsection{HTTP}
  \begin{itemize}
    \item HyperText Transfer Protocol
    \item Client/server model
    \item RFC 1945 (HTTP 1.0), RFC 2616 (HTTP 1.1)
    \item Over TCP
  \end{itemize}
  \subsubsection{Persistent HTTP}
  \begin{itemize}
    \item Multiple objects can be sent over single TCP connection
    \item \textbf{Persistent with pipelining}: client may send requests as soon as it encounters a referenced object -- as little as 1RTT for all referenced objects.
  \end{itemize}
  \subsubsection{Non-Persistent HTTP}
  \begin{itemize}
    \item At most 1 object sent over a TCP connection
    \item Requires 2 RTTs per object
    \item Response time $= 2\times RTT + $ file transmission time
  \end{itemize}
  Refer to slides for the rest of HTTP

  \subsection{DNS}
  \begin{itemize}
    \item \textbf{Distributed, Hierarchical Database}
    \item \textbf{Root Server}: answers requests for records in the root zone by returning a list of the authoritative name servers for the appropriate TLD
    \item \textbf{DNS Caching}: based on TTL
    \item Runs over \textbf{UDP}
  \end{itemize}
  \subsubsection{Resource Records (RR)}
  Stores mapping between hostnames and IP addresses, 4-tuple \texttt{(name, value, type, ttl)}
  \begin{itemize}
    \item type = A, \texttt{name} is hostname, \texttt{value} is IP address
    \item type = NS, \texttt{name} is domain, \texttt{value} is hostname of authoritative name server for the domain
    \item type = CNAME, \texttt{name} is alias for some canonical name, \texttt{value} is the canonical name
    \item type = MX, \texttt{value} is the name of mail server assoc with \texttt{name}
  \end{itemize}
  \subsubsection{DNS Name Resolution}
  \begin{itemize}
    \item \textbf{Iterative query}: Local DNS server makes DNS requests one by one in the hierarchy
    \item \textbf{Recursive query} (rarely used): each server in the hierarchy asks one server higher in the hierarchy
  \end{itemize}

  \subsection{Socket Programming}
  \begin{itemize}
    \item \textbf{IP address} is used to identify a host device
    \item \textbf{Process}: program running within a host, identified by \texttt{(IP address :: uint32, port number :: uint16)}
    \item \textbf{Socket}: the software interface between app processes and transport layer protocols
  \end{itemize}
  \subsubsection{UDP}
  \begin{itemize}
    \item No ``connection'' between client and server.
    \item Sender explicitly attaches destination IP address and port number to each packet
    \item Receiver extracts sender IP address and port number from the received packet
  \end{itemize}
  \subsubsection{TCP}
  \begin{itemize}
    \item When client creates socket, client TCP establishes a copnnection to server TCP.
    \item When contacted by client, server TCP creates a new socket for server process to communicate with that client
    \item Allows server to talk with multiple clients individually.
    \item Communicates as if there is a pipe between 2 processes, sending process doesn't need to attach a destination IP address and port number in each sending attempt.
  \end{itemize}

  \section{Transport Layer}
  \subsection{Transport-layer Services}
  \begin{itemize}
    \item Sender: Breaks app messages into \textbf{segments}, passes them to network layer
    \item Receiver: Reassembles segments into message, passes it to app layer
    \item Packet switches in between: only check destination IP address to decide routing
    \item Each IP datagram contains source and dest IP addresses
  \end{itemize}
  \subsection{Connectionless Transport: UDP}
  \begin{itemize}
    \item UDP addes very little on top of IP
          \begin{itemize}
            \item Multiplexing at sender
            \item Demultiplexing at receiver
            \item Checksum
          \end{itemize}
    \item UDP transmission is unreliable, often used by (loss tolerant \& rate sensitive apps)
  \end{itemize}
  \subsubsection{Connectionless De-multiplexing}
  When UDP receiver receives a UDP segment:
  \begin{itemize}
    \item Check destination port number in segment, and direct that segment to the socket with that port number.
  \end{itemize}
  \subsubsection{UDP Header}
  16 bits each for: source port number, dest port number, length, checksum
  \subsubsection{UDP Checksum}
  \begin{itemize}
    \item Treat segment as sequence of 16-bit integers
    \item Apply binary addition, wraparound carry added to the result
    \item Compute 1's complement to get the checksum
  \end{itemize}
  \subsection{Principles of Reliable Data Transport}
  Refer to slides for rdt example protocols
  \subsection{Connection-oriented Transport: TCP}
  \begin{itemize}
    \item \textbf{Point-to-point}: 1 sender, 1 receiver
    \item \textbf{Connection-oriented}: handshake before sending app data
    \item \textbf{Full duplex service}: bi-directional data flow in the same connection
    \item \textbf{Reliable, in-order byte stream}: sequence numbers to label bytes
  \end{itemize}
  \subsubsection{Connection-oriented de-mux}
  A TCP connection/socket is identified by 4-tuple \texttt{(srcIPAddr, srcPort, destIPAddr, destPort)}
  \subsubsection{TCP: buffers and Segments}
  \begin{itemize}
    \item two buffers, send and receive, are created after handshaking at both sides
    \item \textbf{Max Segment Size (MSS)}: typically 1460 bytes, max app-layer data one TCP segment can carry.
  \end{itemize}
  \subsubsection{TCP Header}
  \begin{verbatim}
|1             16            32|
| sourcePort#      destPort#   |
|       sequence number        |
|    acknowledgement number    |
|                              |
|  checksum    |               |
  \end{verbatim}
  \begin{itemize}
    \item \textbf{Sequence Number}: byte number of the first byte of data in a segment
    \item \textbf{ACK number}: sequence number of the next byte of data expected by the receiver
    \item \textbf{Cumulative ACK}: TCP ACKs up to the first missing byte in the tream
  \end{itemize}
  \subsubsection{TCP ACK Generation, Timeout Value, Fast Retransmission}
  Refer to Lecture 4,5 Slide 66
\end{multicols*}
\end{document}
