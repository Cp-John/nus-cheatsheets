\documentclass[10pt,landscape,a4paper]{article}
%\usepackage[utf8]{inputenc}
%\usepackage[ngerman]{babel}
\usepackage[normalem]{ulem}
\usepackage{tikz}
\usetikzlibrary{shapes,positioning,arrows,fit,calc,graphs,graphs.standard}
\usepackage[nosf]{kpfonts}
\usepackage[t1]{sourcesanspro}
%\usepackage[lf]{MyriadPro}
%\usepackage[lf,minionint]{MinionPro}
\usepackage{multicol}
\usepackage{wrapfig}
\usepackage[top=0mm,bottom=1mm,left=0mm,right=1mm]{geometry}
\usepackage[framemethod=tikz]{mdframed}
\usepackage{microtype}
%\usepackage{physics}
\usepackage{tabularx}
\usepackage{hhline}
\usepackage{makecell}
\usepackage{mathtools}

\usepackage{listings}

\DeclarePairedDelimiter{\ceil}{\lceil}{\rceil}

\newcommand\codeblue[1]{\textcolor{blue}{\code{#1}}}

\usepackage{lastpage}
\usepackage{datetime}
\yyyymmdddate
\renewcommand{\dateseparator}{-}
\let\bar\overline

\definecolor{myblue}{cmyk}{1,.72,0,.38}

\def\firstcircle{(0,0) circle (1.5cm)}
\def\secondcircle{(0:2cm) circle (1.5cm)}

\colorlet{circle edge}{myblue}
\colorlet{circle area}{myblue!5}

\tikzset{filled/.style={fill=circle area, draw=circle edge, thick},
outline/.style={draw=circle edge, thick}}

\pgfdeclarelayer{background}
\pgfsetlayers{background,main}

%\everymath\expandafter{\the\everymath \color{myblue}}
%\everydisplay\expandafter{\the\everydisplay \color{myblue}}


\renewcommand{\baselinestretch}{.8}
\pagestyle{empty}

\global\mdfdefinestyle{header}{%
  linecolor=gray,linewidth=1pt,%
  leftmargin=0mm,rightmargin=0mm,skipbelow=0mm,skipabove=0mm,
}

\newcommand{\header}{
  \begin{mdframed}[style=header]
    \footnotesize
    \sffamily
    CS3245 Midterms Cheatsheet v1.0 (\today)\\
    by~Julius Putra Tanu Setiaji,~page~\thepage~of~\pageref{LastPage}
  \end{mdframed}
}

\let\counterwithout\relax
\let\counterwithin\relax
\usepackage{chngcntr}

\usepackage{verbatim}

\usepackage{etoolbox}
\makeatletter
\preto{\@verbatim}{\topsep=0pt \partopsep=0pt }
\makeatother

\counterwithin*{equation}{section}
\counterwithin*{equation}{subsection}
\usepackage{enumitem}
\newlist{legal}{enumerate}{10}
\setlist[legal]{label*=\arabic*.,leftmargin=2.5mm}
\setlist[itemize]{leftmargin=3mm}
\setlist[enumerate]{leftmargin=3.5mm}
\setlist{nosep}
\usepackage{minted}

\def\code#1{\texttt{#1}}

\newenvironment{descitemize} % a mixture of description and itemize
{\begin{description}[leftmargin=*,before=\let\makelabel\descitemlabel]}
{\end{description}}

\newcommand{\descitemlabel}[1]{%
  \textbullet\ \textbf{#1}%
}
\makeatletter



\renewcommand{\section}{\@startsection{section}{1}{0mm}%
  {.2ex}%
  {.2ex}%x
{\color{myblue}\sffamily\small\bfseries}}
\renewcommand{\subsection}{\@startsection{subsection}{1}{0mm}%
  {.2ex}%
  {.2ex}%x
{\sffamily\bfseries}}
\renewcommand{\subsubsection}{\@startsection{subsubsection}{1}{0mm}%
  {.2ex}%
  {.2ex}%x
{\rmfamily\bfseries}}



\def\multi@column@out{%
  \ifnum\outputpenalty <-\@M
    \speci@ls \else
  \ifvoid\colbreak@box\else
    \mult@info\@ne{Re-adding forced
    break(s) for splitting}%
    \setbox\@cclv\vbox{%
      \unvbox\colbreak@box
    \penalty-\@Mv\unvbox\@cclv}%
  \fi
  \splittopskip\topskip
  \splitmaxdepth\maxdepth
  \dimen@\@colroom
  \divide\skip\footins\col@number
  \ifvoid\footins \else
    \leave@mult@footins
  \fi
  \let\ifshr@kingsaved\ifshr@king
    \ifvbox \@kludgeins
      \advance \dimen@ -\ht\@kludgeins
      \ifdim \wd\@kludgeins>\z@
        \shr@nkingtrue
      \fi
    \fi
    \process@cols\mult@gfirstbox{%
      %%%%% START CHANGE
      \ifnum\count@=\numexpr\mult@rightbox+2\relax
        \setbox\count@\vsplit\@cclv to \dimexpr \dimen@-1cm\relax
        \setbox\count@\vbox to \dimen@{\vbox to 1cm{\header}\unvbox\count@\vss}%
      \else
        \setbox\count@\vsplit\@cclv to \dimen@
      \fi
      %%%%% END CHANGE
      \set@keptmarks
      \setbox\count@
      \vbox to\dimen@
      {\unvbox\count@
        \remove@discardable@items
    \ifshr@nking\vfill\fi}%
    }%
    \setbox\mult@rightbox
    \vsplit\@cclv to\dimen@
    \set@keptmarks
    \setbox\mult@rightbox\vbox to\dimen@
    {\unvbox\mult@rightbox
      \remove@discardable@items
  \ifshr@nking\vfill\fi}%
    \let\ifshr@king\ifshr@kingsaved
  \ifvoid\@cclv \else
    \unvbox\@cclv
    \ifnum\outputpenalty=\@M
  \else
    \penalty\outputpenalty
  \fi
  \ifvoid\footins\else
    \PackageWarning{multicol}%
    {I moved some lines to
      the next page.\MessageBreak
      Footnotes on page
    \thepage\space might be wrong}%
  \fi
  \ifnum \c@tracingmulticols>\thr@@
\hrule\allowbreak \fi
  \fi
  \ifx\@empty\kept@firstmark
    \let\firstmark\kept@topmark
    \let\botmark\kept@topmark
  \else
    \let\firstmark\kept@firstmark
    \let\botmark\kept@botmark
  \fi
  \let\topmark\kept@topmark
  \mult@info\tw@
  {Use kept top mark:\MessageBreak
    \meaning\kept@topmark
    \MessageBreak
    Use kept first mark:\MessageBreak
    \meaning\kept@firstmark
    \MessageBreak
    Use kept bot mark:\MessageBreak
    \meaning\kept@botmark
    \MessageBreak
    Produce first mark:\MessageBreak
    \meaning\firstmark
    \MessageBreak
    Produce bot mark:\MessageBreak
    \meaning\botmark
  \@gobbletwo}%
  \setbox\@cclv\vbox{\unvbox\partial@page
  \page@sofar}%
  \@makecol\@outputpage
  \global\let\kept@topmark\botmark
  \global\let\kept@firstmark\@empty
  \global\let\kept@botmark\@empty
  \mult@info\tw@
  {(Re)Init top mark:\MessageBreak
    \meaning\kept@topmark
  \@gobbletwo}%
  \global\@colroom\@colht
  \global \@mparbottom \z@
  \process@deferreds
\@whilesw\if@fcolmade\fi{\@outputpage
    \global\@colroom\@colht
  \process@deferreds}%
  \mult@info\@ne
  {Colroom:\MessageBreak
    \the\@colht\space
    after float space removed
  = \the\@colroom \@gobble}%
  \set@mult@vsize \global
  \fi}
  \global\let\tikz@ensure@dollar@catcode=\relax

  \def\mathcolor#1#{\@mathcolor{#1}}
  \def\@mathcolor#1#2#3{%
    \protect\leavevmode
    \begingroup
    \color#1{#2}#3%
    \endgroup
  }

  \makeatother
  \setlength{\parindent}{0pt}

  \setminted{tabsize=2, breaklines}
  % Remove belowskip of minted
  \setlength\partopsep{-\topsep}


  \newcolumntype{a}{>{\hsize=1.5\hsize}X}
  \newcolumntype{b}{>{\hsize=.25\hsize}X}

  \setlength\columnsep{1.5pt}
  \setlength\columnseprule{0.1pt}
  
\begin{document}
\setlength{\abovedisplayskip}{0pt}
\setlength{\belowdisplayskip}{0pt}


\scriptsize
\begin{multicols*}{3}
  \raggedcolumns
  \section{Language Models}
  \subsection{Ngram model}
  \begin{itemize}
    \item Unigram model (n = 1) does not model word order (just a "bag of words").
    \item Can be defined as bags of phrases of length n.
    \item OR a model that can predict a current word from the $(n - 1)$ previous context words, e.g. in a pentagram model $P(?? \mid \text{"Please turn off your hand"})$
    \item Longer n-gram models are exponentially more costly to construct. Let $\lvert V \rvert$ = size of vocab in a language. In unigram, store probabilities for all $\lvert V \rvert$ words. In bigram, need to store all $\lvert V \rvert \times \lvert V \rvert$ ordered length 2 phrases.
  \end{itemize}
  \subsubsection{Markov Assumption}
  \begin{itemize}
    \item Presumption that the future behaviour of a dynamical system only depends on its recent history.
    \item A \textbf{kth-order Markov model}, the next state only depends on the $k$ most recent states.
    \item Hence, an N-gram model = (N - 1)-order Markov model.
  \end{itemize}
  \subsubsection{Add 1 smoothing}
  \begin{itemize}
    \item Problem with zero probability.
    \item Add 1 count to all entries in the LM, including those that are not seen.
    \item Not used in practice, but most basic to understand.
    \item Add-one or add-k smoothing represents a uniform belief in all events (i.e. possible ngrams) as it assigns a simple $\frac{1}{n}$ probability to all events.
    \item Applied to:
    \begin{itemize}
      \item \textbf{small} no of observations, \textbf{large} vocab space: \textbf{poor} choice (shifting the probability mass quite drastically, given small no observation but large no of possible outcomes (vocab space). Extreme example: English alphabet, only observe "a", then P(a) = 1.0, but with add-one smoothing, P(a) = $\frac{2}{27} = 0.074$)
      \item \textbf{large} no of observations, \textbf{small} vocab space: \textbf{good} choice (only shift a small probability mass while maintaining the important guarantee of non-zero probability for all possible events.)
    \end{itemize}
  \end{itemize}

  \section{Boolean Retrieval}
  \subsection{Term-document Incidence}
  \begin{tabular}{c | c c c}
      & Harriet & Othello & Macbeth \\ \hline
    Antony & 0 & 0 & 1 \\
    Brutus & 1 & 0 & 0 \\
    Caesar & 1 & 1 & 1 \\
  \end{tabular}
  \begin{itemize}
    \item A vector of 1s and 0s for each term, whether they appear in the document.
    \item Query processing: take the vectors and perform bitwise AND (complement for NOT)
    \item Disadvantage: Big, but sparse matrix.
    \item Better to just store the position of the 1's.
  \end{itemize}
  \subsection{Inverted Index}
  \begin{verbatim}
Brutus    |8|  => 1, 2, 4, 11, 31, 45, 173, 174
Caesar    |8|  => 1, 2, 4, 5, 6, 16, 57, 134
Calpurnia |4|  => 2, 31, 54, 101
  \end{verbatim}
  \begin{itemize}
    \item For each term $t$, we store a list of docIDs that contain $t$
    \item Requires variable-size postings list. (on disk, continuous run of postings. In memory, linked list or variable length arrays)
    \item docID inside each term is sorted.
    \item Construction process:
    \begin{enumerate}
      \item \textbf{Tokenise} into a sequence of (modified token, document ID) pairs.
      \item \textbf{Sort} by terms, and then by docID
      \item Multiple term entries in a single document are merged, split into \textbf{dictionary} and \textbf{postings}, and store \textbf{document frequency}.
    \end{enumerate}
    \item Query processing AND:
    \begin{itemize}
      \item Locate both terms in the dictionary, and merge the two postings.
      \item This is $O(n + m)$ if docID are sorted.
      \item Optimisation: process in order of increasing frequency
    \end{itemize}
  \end{itemize}
  \subsection{Boolean Queries: Exact match}
  \begin{itemize}
    \item Operator precedence: NOT, AND, OR
  \end{itemize}
  \section{Postings lists and choosing terms}
  \subsection{Skip pointers}
  \begin{itemize}
    \item Done at indexing time.
    \item Tradeoff:
    \begin{itemize}
      \item \textbf{More skips} -> shorter skip spans -> more likely to skip, bot lots of comparisons.
      \item \textbf{Fewer skips} -> few pointer comparison, but long skip spans -> few successful skips.
    \end{itemize}
    \item \textbf{Simple heuristic}: for postings of length $L$, use $\sqrt{L}$ evenly-placed skip pointers.
    \begin{itemize}
      \item Ignores the distribution of query terms (e.g. 1, 2, 3, 4, 5, 6, 10, 20, 40, 50 has skip pointers 1, 4, 10, 50)
      \item Easy if index is relatively static, harder if $L$ keeps on changing because of updates.
      \item Used to help , but with modern hardware it may not unless memory-based (because IO cost of loading a bigger postings list can outweigh the gains from quicker in-memory merging)
    \end{itemize}
  \end{itemize}
  \subsection{Phrase Queries}
  \subsubsection{Biword indexes}
  \begin{itemize}
    \item Index every consecutive pair of terms as a phrase.
    \item 2-word phrase query processing is now immediate.
    \item Longer phrase processed as booleqn query on biwords. But can have false positives (without the docs, cannot make sure that the biwords appear in that order exactly)
    \item \textbf{Extended Biwords}
    \begin{itemize}
      \item Parse the text and perform part-of-speech-tagging.
      \item Bucket the terms into Nouns (N) and articles/prepositions (X)
      \item Call any string of terms of the form \texttt{NX*N} an extended biword.
      \item Each extended biword is now a term in the dictionary
      \item E.g. "catcher in the rye" (NXXN) => "catcher rye"
      \item Disadvantages: False positives, Index blowup (bigger dictionary)
    \end{itemize}
  \end{itemize}
  \subsubsection{Positional indexes}
  \begin{verbatim}
  <term, no of docs containing term;
  doc1: position1, position2, ...;
  doc2: position1, position2, ...;
  ...>
  \end{verbatim}
  \begin{itemize}
    \item In the postings, store, for each \textbf{term} the position(s) that the term appear
    \item For phrase queries, use a merge algorithm recursively at the document level.
    \item Query processing:
    \begin{itemize}
      \item \textbf{Extract} the inverted index entries for each distinct term
      \item \textbf{Merge} their \texttt{doc:position} lists to enumerate all positions with the given phrase, e.g. searching for "to be", must look for documents that has "to" in which the position of "be" is exactly afterwards
      \item Same strategy for proximity queries.
    \end{itemize}
    \item Expands postings storage substantially. But standard because of the power and usefulness.
    \item Rules of thumb: A positional index is 2-4x larger as non-posiitonal index, and around 35-50\% of the original text (for English-like languages)
    \item Can be combined with biwords (e.g. for oft-queried phrases, inefficient to keep on merging positional postings lists)
  \end{itemize}
  \subsection{Token and terms}
  \subsubsection{Extracting text}
  Granularity: what unit? file/email/group of files?
  \subsubsection{Tokenization}
  \begin{itemize}
    \item Token = instance of a sequence of characters grouped together as a useful semantic unit
    \item Issues:
    \begin{itemize}
      \item Handling apostrophe, hyphens, spaces in proper names, numbers, dates.
      \item Language issues: e.g. in French, l'ensemble and un ensemble, the non-segmented German noun compounds, Chinese and Japanese have no spaces between words
      \item LTR, RTL langauges.
      \item \textbf{Stop words}: exclude words from stop list: most common words from the dictionary (little semantic content), but trend is away from doing this (good compression and query optimization can mitigate the size)
    \end{itemize}
  \end{itemize}
  \subsubsection{Normalization}
  \begin{itemize}
    \item Removing dots from abbreviations, hyphens, diacritics.
    \item Normalization of date forms, (Japanese) kana vs kanji
    \item \textbf{Case folding}
    \begin{itemize}
      \item Reduce all letters to lower case
      \item Disadvantage: C.A.T. vs cat, MIT vs mit (German)
    \end{itemize}
    \item Alternative to equivalence classing is to do asymmetric expansion (e.g. window => window, windows => Windows, windows, window; Windows => Windows)
    \item \textbf{Thesauri}: handle synonyms and homonyms (car = automobile, color = colour)
  \end{itemize}
  \subsubsection{Lemmatization}
  \begin{itemize}
    \item Reduce inflectional/variant forms to base form with the proper way (linguistically)
    \item A most modest benefit for retrieval
    \item For English, mixed results, but definitely useful for Spanish, German, Finnish (30\% perf gain for Finnish)
  \end{itemize}
  \subsubsection{Stemming}
  \begin{itemize}
    \item Reduce terms to their roots with crude affix chopping (e.g. automate, automatic, automation => automat)
    \item \textbf{Porter's algorithm}: most common for stemming English
    \item Other stemmers, e.g. Lovins stemmer (single-pass, longest suffix removal, about 250 rules)
  \end{itemize}
  \section{Dictionaries and tolerant retrieval}
  \subsection{Hash Table}
  \begin{itemize}
    \item Pros: Lookup is faster than for a tree: $O(1)$
    \item Cons:
    \begin{itemize}
      \item No easy way to find minor variants
      \item No prefix search
      \item If vocab keeps growing, need to occasionally perform expensive operation of rehasing \textbf{everything}
    \end{itemize}
  \end{itemize}
  \subsection{Tree}
  \begin{itemize}
    \item Requires standard ordering of characters and strings: lexicographical ordering
    \item Pros: solves the prefix problem (\texttt{hyp*})
    \item Cons:
    \begin{itemize}
      \item Slower: $O(\log M)$ and requires a \textbf{balanced} tree
      \item Rebalancing binary trees is expensive (B-trees mitigate the rebalancing problem)
    \end{itemize}
  \end{itemize}
  \subsubsection{Binary tree}
  Simplest, whereby each branch divides by two, and only the leaves store the words.
  \subsubsection{B-tree}
  Every internal node has a number of children has a number of children in the interval $[a, b]$ where $a, b$ are appropriate natural numbers.
  \subsection{Wildcard queries}
  \begin{itemize}
    \item With B-tree, simple for \texttt{mon*}
    \item \texttt{*mon}: maintain an additional B-tree for the terms reversed.
    \item \texttt{A*B}: intersect \texttt{A*} and \texttt{*B}
    \item However, still have to look up the postings for each enumerated term (expensive)
  \end{itemize}
  \subsection{Permuterm Index}
  \begin{itemize}
    \item For "hello", index under: "hello\$", "ello\$h", "llo\$he", "lo\$hel", "o\$hell", "\$hello"
    \item Queries:
    \begin{itemize}
      \item \texttt{X} lookup on \texttt{X\$}
      \item \texttt{*X} lookup on \texttt{X\$*}
      \item \texttt{X*Y} lookup on \texttt{Y\$X*}
      \item \texttt{X*} lookup on \texttt{\$X*}
      \item \texttt{*X*} lookup on \texttt{X*}
      \item \texttt{X*Y*Z} lookup on \texttt{\$X*} and \texttt{Z\$*}, AND them and find which of them contains \texttt{Y}
    \end{itemize}
    \item Problem: lexicon size blows up, proportional to average word length
  \end{itemize}
  \subsection{Bigram (k-gram) indexes}
  \begin{itemize}
    \item Enumerate all k-grams (sequence of k chars) occuring in any term (per letter)
    \item Maintain a second inverted index from bigrams to dictionary terms that match each bigram
    \item E.g. \texttt{mon*} = \texttt{\$m} AND \texttt{mo} AND \texttt{on} (usually not useful to add \texttt{n\$})
    \item Prone to false positive, e.g. in this case moon, thus must post filter.
    \item Fast, space efficient (comapred to permuterm)
  \end{itemize}
  \subsection{Spelling Correction}
  \subsubsection{Isolated word correction}
  \begin{itemize}
    \item Given query, enumerate all character sequences within a preset (weighted) edit distance (e.g. 2), intersect this set with list of correct words.
    \item Alternatively:
    \begin{itemize}
      \item Look up all possible corrections in our inverted index and return all docs (slow)
      \item We can run with a single most likely correction
    \end{itemize}
    \item However, computing edit distance to every dictionary term is expensive and slow. Alternative: use ngram overlap
  \end{itemize}
  How to define closest?
  \begin{enumerate}
    \item \textbf{Edit distance (Levenshtein distance)}
    \begin{itemize}
      \item Given two strings, the minimum no of ops to convert one to the other.
      \item Ops: insert, delete, replace (optionally transposition)
      \item \textbf{Wagner–Fischer algorithm}:\\
      Create a matrix whereby $E(i, j) = min \{ E(i, j - 1) + 1, E(i - 1, j) + 1, E(i - 1, j - 1) + m \}$ where $m = 1$ if $P_i \neq T_j$, $0$ otherwise
    \end{itemize}
    \item \textbf{Weighted edit distance}
    \begin{itemize}
      \item As above, but the weight of an operation depends on the characters involved, e.g. keyboard errors, m is more likely to be mistyped as n than as q
      \item Requires a weighted matrix as input
    \end{itemize}
    \item \textbf{Ngram overlap}
    \begin{itemize}
      \item Enumerate all ngrams in the query strings as well as in the lexicon
      \item Use the ngram index (from wildcard search) to retrieve all lexicon terms matching any of the query ngrams
      \item Threshold by no of matching ngrams (weight by e.g. keyboard layout, assume initial letter correct, etc.)
      \item \textbf{Jaccard coefficient} = $\frac{X\cap Y}{X \cup Y}$
      \begin{itemize}
        \item 1 when X and Y have the same elements, 0 when they are disjoint
        \item Always assigns a number between 0 and 1.
        \item Use a threshold to decide if you have a match, e.g. above 0.8 is a match
        \item To calculate the union, can use $n(X) + n(Y) - n(X \cap Y)$
      \end{itemize}
    \end{itemize}
  \end{enumerate}
  \subsubsection{Context-Sensitive}
  \begin{itemize}
    \item Need surrounding context
    \item Retrieve dictionary terms close (in weighted edit distance) to each query term
    \item Try all possible resulting phrases with one word corrected at a time
    \item \textbf{Hit-based spelling correction}: suggest the alternative that has lots of hits
    \item Alternative: break phrase queries into conjunctions of biwords, look for ones that need only one term corrected, enumerate phrase matches and rank them.
  \end{itemize}
  \subsection{Soundex}
  \begin{itemize}
    \item Class of heuristics to expand a query into phonetic equivalents (language specific)
    \item Invented for the US cencus
    \item Turn every token to be indexed into a 4-character reduced form
    \item Build and search an index on the reduced form
    \item Algorithm: refer to lecture notes
    \item Not very useful for general IR, okay for high recall tasks, though biased to names of certain nationalities.
  \end{itemize}
\end{multicols*}
\end{document}
